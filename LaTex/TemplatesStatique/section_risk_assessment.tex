% Modular Risk Assessment Section
% This file contains ONLY the risk assessment content
% Usage: % Modular Risk Assessment Section
% This file contains ONLY the risk assessment content
% Usage: % Modular Risk Assessment Section
% This file contains ONLY the risk assessment content
% Usage: % Modular Risk Assessment Section
% This file contains ONLY the risk assessment content
% Usage: \input{section_risk_assessment} in your main document

\section{Analyse des Risques}

\subsection{Matrice d'Évaluation des Risques}

\begin{table}[H]
\centering
\caption{Évaluation des Risques Principaux}
\label{tab:risk-matrix}
\begin{tabular}{|p{0.25\textwidth}|c|c|c|p{0.3\textwidth}|}
\hline
\rowcolor{DollaramaGreen!30}
\textbf{\color{white}Risque} & 
\textbf{\color{white}Probabilité} & 
\textbf{\color{white}Impact} & 
\textbf{\color{white}Score} & 
\textbf{\color{white}Mitigation} \\
\hline
[Risque technique] & \cellcolor{yellow!50}\textbf{M} & \cellcolor{yellow!50}\textbf{M} & \cellcolor{yellow!50}\textbf{6} & [Stratégie de mitigation] \\
\hline
\rowcolor{gray!10}
[Risque financier] & \cellcolor{green!30}\textbf{F} & \cellcolor{red!30}\textbf{É} & \cellcolor{yellow!50}\textbf{6} & [Stratégie de mitigation] \\
\hline
[Risque opérationnel] & \cellcolor{green!30}\textbf{F} & \cellcolor{yellow!50}\textbf{M} & \cellcolor{green!30}\textbf{3} & [Stratégie de mitigation] \\
\hline
\rowcolor{gray!10}
[Risque de délai] & \cellcolor{yellow!50}\textbf{M} & \cellcolor{yellow!50}\textbf{M} & \cellcolor{yellow!50}\textbf{6} & [Stratégie de mitigation] \\
\hline
[Risque organisationnel] & \cellcolor{green!30}\textbf{F} & \cellcolor{red!30}\textbf{É} & \cellcolor{yellow!50}\textbf{6} & [Stratégie de mitigation] \\
\hline
[Risque externe] & \cellcolor{green!30}\textbf{F} & \cellcolor{yellow!50}\textbf{M} & \cellcolor{green!30}\textbf{3} & [Stratégie de mitigation] \\
\hline
\end{tabular}
\end{table}

\textbf{Légende:} F = Faible, M = Moyen, É = Élevé | Score = Probabilité × Impact (1-3 scale)

% Risk Heat Map
\begin{table}[H]
\centering
\caption{Matrice de Chaleur des Risques}
\label{tab:risk-heatmap}
\begin{tabular}{|l|c|c|c|c|}
\hline
\rowcolor{DollaramaGreen!30}
\textbf{\color{white}Impact/Probabilité} & 
\textbf{\color{white}Très Faible} & 
\textbf{\color{white}Faible} & 
\textbf{\color{white}Moyenne} & 
\textbf{\color{white}Élevée} \\
\hline
\textbf{Critique} & \cellcolor{yellow!50}4 & \cellcolor{orange!50}8 & \cellcolor{red!50}12 & \cellcolor{red!70}16 \\
\hline
\textbf{Majeur} & \cellcolor{green!50}3 & \cellcolor{yellow!50}6 & \cellcolor{orange!50}9 & \cellcolor{red!50}12 \\
\hline
\textbf{Modéré} & \cellcolor{green!30}2 & \cellcolor{green!50}4 & \cellcolor{yellow!50}6 & \cellcolor{orange!50}8 \\
\hline
\textbf{Mineur} & \cellcolor{green!20}1 & \cellcolor{green!30}2 & \cellcolor{green!50}3 & \cellcolor{yellow!50}4 \\
\hline
\end{tabular}
\end{table}

\subsection{Analyse Détaillée des Risques}

\subsubsection{Risques Techniques}

\preventionbox{
\textbf{STRATÉGIES DE PRÉVENTION - RISQUES TECHNIQUES}

Cette section doit détailler les mesures préventives pour chaque risque technique identifié:
\begin{itemize}
\item \textbf{[Risque 1]:} Description de la stratégie de prévention
\item \textbf{[Risque 2]:} Description de la stratégie de prévention
\item \textbf{[Risque 3]:} Description de la stratégie de prévention
\end{itemize}

Incluez des actions spécifiques, des responsables et des échéances pour chaque mesure.
}

\subsubsection{Risques Opérationnels}

\contingencybox{
\textbf{PLANS DE CONTINGENCE - RISQUES OPÉRATIONNELS}

Décrivez les plans alternatifs si les risques se matérialisent:
\begin{itemize}
\item \textbf{[Plan A]:} Actions à prendre si [condition]
\item \textbf{[Plan B]:} Alternative si le Plan A échoue
\item \textbf{[Ressources de secours]:} Équipes/budget disponibles en urgence
\end{itemize}

Assurez-vous que chaque plan a un déclencheur clair et des responsables désignés.
}

\subsubsection{Risques Financiers}

\recoverybox{
\textbf{STRATÉGIES DE RÉCUPÉRATION - RISQUES FINANCIERS}

Plans pour minimiser l'impact financier si les risques se réalisent:
\begin{itemize}
\item \textbf{[Dépassement budget]:} Actions correctives et sources de financement alternatives
\item \textbf{[Retard ROI]:} Mesures d'accélération des bénéfices
\item \textbf{[Coûts cachés]:} Processus d'identification et de contrôle
\end{itemize}

Quantifiez l'impact potentiel et les coûts de récupération.
}

\subsection{Tableau de Bord des Risques}

\vspace{1em}
\begin{center}
\fcolorbox{DollaramaGreen}{DollaramaGreen!5}{%
\begin{minipage}{0.9\textwidth}
\vspace{0.5em}
\begin{center}
{\Large\bfseries\color{DollaramaGreen} TABLEAU DE BORD DES RISQUES}
\end{center}
\vspace{0.5em}

\begin{tabular}{p{0.22\textwidth}p{0.22\textwidth}p{0.22\textwidth}p{0.22\textwidth}}
\centering
\fcolorbox{green}{white}{%
\begin{minipage}{0.2\textwidth}
\centering
\textbf{Risques Faibles}\\
\vspace{0.2em}
{\Large\color{green}[X]}
\end{minipage}
} & 
\centering
\fcolorbox{orange}{white}{%
\begin{minipage}{0.2\textwidth}
\centering
\textbf{Risques Moyens}\\
\vspace{0.2em}
{\Large\color{orange}[Y]}
\end{minipage}
} &
\centering
\fcolorbox{red}{white}{%
\begin{minipage}{0.2\textwidth}
\centering
\textbf{Risques Élevés}\\
\vspace{0.2em}
{\Large\color{red}[Z]}
\end{minipage}
} &
\centering
\fcolorbox{DollaramaGreen}{white}{%
\begin{minipage}{0.2\textwidth}
\centering
\textbf{Mitigations Actives}\\
\vspace{0.2em}
{\Large\color{DollaramaGreen}[N]}
\end{minipage}
}
\end{tabular}

\vspace{0.5em}
\end{minipage}
}
\end{center}
\vspace{1em}

\subsection{Plan de Gestion des Risques}

\begin{longtable}{|p{0.2\textwidth}|p{0.3\textwidth}|p{0.15\textwidth}|p{0.15\textwidth}|p{0.15\textwidth}|}
\caption{Plan de Suivi des Risques} \label{tab:risk-management} \\
\hline
\rowcolor{DollaramaGreen!30}
\textbf{\color{white}Risque} & 
\textbf{\color{white}Action de Mitigation} & 
\textbf{\color{white}Responsable} & 
\textbf{\color{white}Échéance} & 
\textbf{\color{white}Statut} \\
\hline
\endfirsthead

\multicolumn{5}{c}%
{{\tablename\ \thetable{} -- Suite de la page précédente}} \\
\hline
\rowcolor{DollaramaGreen!30}
\textbf{\color{white}Risque} & 
\textbf{\color{white}Action de Mitigation} & 
\textbf{\color{white}Responsable} & 
\textbf{\color{white}Échéance} & 
\textbf{\color{white}Statut} \\
\hline
\endhead

\hline \multicolumn{5}{|r|}{{Suite sur la page suivante}} \\ \hline
\endfoot

\hline
\endlastfoot

[Risque 1] & [Action détaillée] & [Nom] & [Date] & \cellcolor{yellow!30}[Statut] \\
\hline
\rowcolor{gray!10}
[Risque 2] & [Action détaillée] & [Nom] & [Date] & \cellcolor{green!30}[Statut] \\
\hline
[Risque 3] & [Action détaillée] & [Nom] & [Date] & \cellcolor{green!30}[Statut] \\
\hline
\rowcolor{gray!10}
[Risque 4] & [Action détaillée] & [Nom] & [Date] & \cellcolor{yellow!30}[Statut] \\
\hline
[Risque 5] & [Action détaillée] & [Nom] & [Date] & \cellcolor{red!30}[Statut] \\
\hline
\rowcolor{gray!10}
[Risque 6] & [Action détaillée] & [Nom] & [Date] & \cellcolor{green!30}[Statut] \\
\hline
\end{longtable}

\textbf{Légende Statuts:} \colorbox{green!30}{Planifié} | \colorbox{yellow!30}{En cours} | \colorbox{red!30}{En retard}

\subsection{Critères d'Escalade}

\warningbox{
\textbf{DÉCLENCHEURS D'ESCALADE IMMÉDIATE:}

Définissez clairement les conditions qui nécessitent une escalade:
\begin{enumerate}
\item [Condition 1 - Ex: Dépassement budgétaire > 10\%]
\item [Condition 2 - Ex: Retard sur jalons critiques > 2 semaines]
\item [Condition 3 - Ex: Problème de sécurité identifié]
\item [Condition 4 - Ex: Résistance organisationnelle majeure]
\item [Condition 5 - Ex: Défaillance technique critique]
\end{enumerate}

\textbf{Processus d'escalade:} [Décrivez qui contacter, dans quel délai, et avec quelle information]
} in your main document

\section{Analyse des Risques}

\subsection{Matrice d'Évaluation des Risques}

\begin{table}[H]
\centering
\caption{Évaluation des Risques Principaux}
\label{tab:risk-matrix}
\begin{tabular}{|p{0.25\textwidth}|c|c|c|p{0.3\textwidth}|}
\hline
\rowcolor{DollaramaGreen!30}
\textbf{\color{white}Risque} & 
\textbf{\color{white}Probabilité} & 
\textbf{\color{white}Impact} & 
\textbf{\color{white}Score} & 
\textbf{\color{white}Mitigation} \\
\hline
[Risque technique] & \cellcolor{yellow!50}\textbf{M} & \cellcolor{yellow!50}\textbf{M} & \cellcolor{yellow!50}\textbf{6} & [Stratégie de mitigation] \\
\hline
\rowcolor{gray!10}
[Risque financier] & \cellcolor{green!30}\textbf{F} & \cellcolor{red!30}\textbf{É} & \cellcolor{yellow!50}\textbf{6} & [Stratégie de mitigation] \\
\hline
[Risque opérationnel] & \cellcolor{green!30}\textbf{F} & \cellcolor{yellow!50}\textbf{M} & \cellcolor{green!30}\textbf{3} & [Stratégie de mitigation] \\
\hline
\rowcolor{gray!10}
[Risque de délai] & \cellcolor{yellow!50}\textbf{M} & \cellcolor{yellow!50}\textbf{M} & \cellcolor{yellow!50}\textbf{6} & [Stratégie de mitigation] \\
\hline
[Risque organisationnel] & \cellcolor{green!30}\textbf{F} & \cellcolor{red!30}\textbf{É} & \cellcolor{yellow!50}\textbf{6} & [Stratégie de mitigation] \\
\hline
[Risque externe] & \cellcolor{green!30}\textbf{F} & \cellcolor{yellow!50}\textbf{M} & \cellcolor{green!30}\textbf{3} & [Stratégie de mitigation] \\
\hline
\end{tabular}
\end{table}

\textbf{Légende:} F = Faible, M = Moyen, É = Élevé | Score = Probabilité × Impact (1-3 scale)

% Risk Heat Map
\begin{table}[H]
\centering
\caption{Matrice de Chaleur des Risques}
\label{tab:risk-heatmap}
\begin{tabular}{|l|c|c|c|c|}
\hline
\rowcolor{DollaramaGreen!30}
\textbf{\color{white}Impact/Probabilité} & 
\textbf{\color{white}Très Faible} & 
\textbf{\color{white}Faible} & 
\textbf{\color{white}Moyenne} & 
\textbf{\color{white}Élevée} \\
\hline
\textbf{Critique} & \cellcolor{yellow!50}4 & \cellcolor{orange!50}8 & \cellcolor{red!50}12 & \cellcolor{red!70}16 \\
\hline
\textbf{Majeur} & \cellcolor{green!50}3 & \cellcolor{yellow!50}6 & \cellcolor{orange!50}9 & \cellcolor{red!50}12 \\
\hline
\textbf{Modéré} & \cellcolor{green!30}2 & \cellcolor{green!50}4 & \cellcolor{yellow!50}6 & \cellcolor{orange!50}8 \\
\hline
\textbf{Mineur} & \cellcolor{green!20}1 & \cellcolor{green!30}2 & \cellcolor{green!50}3 & \cellcolor{yellow!50}4 \\
\hline
\end{tabular}
\end{table}

\subsection{Analyse Détaillée des Risques}

\subsubsection{Risques Techniques}

\preventionbox{
\textbf{STRATÉGIES DE PRÉVENTION - RISQUES TECHNIQUES}

Cette section doit détailler les mesures préventives pour chaque risque technique identifié:
\begin{itemize}
\item \textbf{[Risque 1]:} Description de la stratégie de prévention
\item \textbf{[Risque 2]:} Description de la stratégie de prévention
\item \textbf{[Risque 3]:} Description de la stratégie de prévention
\end{itemize}

Incluez des actions spécifiques, des responsables et des échéances pour chaque mesure.
}

\subsubsection{Risques Opérationnels}

\contingencybox{
\textbf{PLANS DE CONTINGENCE - RISQUES OPÉRATIONNELS}

Décrivez les plans alternatifs si les risques se matérialisent:
\begin{itemize}
\item \textbf{[Plan A]:} Actions à prendre si [condition]
\item \textbf{[Plan B]:} Alternative si le Plan A échoue
\item \textbf{[Ressources de secours]:} Équipes/budget disponibles en urgence
\end{itemize}

Assurez-vous que chaque plan a un déclencheur clair et des responsables désignés.
}

\subsubsection{Risques Financiers}

\recoverybox{
\textbf{STRATÉGIES DE RÉCUPÉRATION - RISQUES FINANCIERS}

Plans pour minimiser l'impact financier si les risques se réalisent:
\begin{itemize}
\item \textbf{[Dépassement budget]:} Actions correctives et sources de financement alternatives
\item \textbf{[Retard ROI]:} Mesures d'accélération des bénéfices
\item \textbf{[Coûts cachés]:} Processus d'identification et de contrôle
\end{itemize}

Quantifiez l'impact potentiel et les coûts de récupération.
}

\subsection{Tableau de Bord des Risques}

\vspace{1em}
\begin{center}
\fcolorbox{DollaramaGreen}{DollaramaGreen!5}{%
\begin{minipage}{0.9\textwidth}
\vspace{0.5em}
\begin{center}
{\Large\bfseries\color{DollaramaGreen} TABLEAU DE BORD DES RISQUES}
\end{center}
\vspace{0.5em}

\begin{tabular}{p{0.22\textwidth}p{0.22\textwidth}p{0.22\textwidth}p{0.22\textwidth}}
\centering
\fcolorbox{green}{white}{%
\begin{minipage}{0.2\textwidth}
\centering
\textbf{Risques Faibles}\\
\vspace{0.2em}
{\Large\color{green}[X]}
\end{minipage}
} & 
\centering
\fcolorbox{orange}{white}{%
\begin{minipage}{0.2\textwidth}
\centering
\textbf{Risques Moyens}\\
\vspace{0.2em}
{\Large\color{orange}[Y]}
\end{minipage}
} &
\centering
\fcolorbox{red}{white}{%
\begin{minipage}{0.2\textwidth}
\centering
\textbf{Risques Élevés}\\
\vspace{0.2em}
{\Large\color{red}[Z]}
\end{minipage}
} &
\centering
\fcolorbox{DollaramaGreen}{white}{%
\begin{minipage}{0.2\textwidth}
\centering
\textbf{Mitigations Actives}\\
\vspace{0.2em}
{\Large\color{DollaramaGreen}[N]}
\end{minipage}
}
\end{tabular}

\vspace{0.5em}
\end{minipage}
}
\end{center}
\vspace{1em}

\subsection{Plan de Gestion des Risques}

\begin{longtable}{|p{0.2\textwidth}|p{0.3\textwidth}|p{0.15\textwidth}|p{0.15\textwidth}|p{0.15\textwidth}|}
\caption{Plan de Suivi des Risques} \label{tab:risk-management} \\
\hline
\rowcolor{DollaramaGreen!30}
\textbf{\color{white}Risque} & 
\textbf{\color{white}Action de Mitigation} & 
\textbf{\color{white}Responsable} & 
\textbf{\color{white}Échéance} & 
\textbf{\color{white}Statut} \\
\hline
\endfirsthead

\multicolumn{5}{c}%
{{\tablename\ \thetable{} -- Suite de la page précédente}} \\
\hline
\rowcolor{DollaramaGreen!30}
\textbf{\color{white}Risque} & 
\textbf{\color{white}Action de Mitigation} & 
\textbf{\color{white}Responsable} & 
\textbf{\color{white}Échéance} & 
\textbf{\color{white}Statut} \\
\hline
\endhead

\hline \multicolumn{5}{|r|}{{Suite sur la page suivante}} \\ \hline
\endfoot

\hline
\endlastfoot

[Risque 1] & [Action détaillée] & [Nom] & [Date] & \cellcolor{yellow!30}[Statut] \\
\hline
\rowcolor{gray!10}
[Risque 2] & [Action détaillée] & [Nom] & [Date] & \cellcolor{green!30}[Statut] \\
\hline
[Risque 3] & [Action détaillée] & [Nom] & [Date] & \cellcolor{green!30}[Statut] \\
\hline
\rowcolor{gray!10}
[Risque 4] & [Action détaillée] & [Nom] & [Date] & \cellcolor{yellow!30}[Statut] \\
\hline
[Risque 5] & [Action détaillée] & [Nom] & [Date] & \cellcolor{red!30}[Statut] \\
\hline
\rowcolor{gray!10}
[Risque 6] & [Action détaillée] & [Nom] & [Date] & \cellcolor{green!30}[Statut] \\
\hline
\end{longtable}

\textbf{Légende Statuts:} \colorbox{green!30}{Planifié} | \colorbox{yellow!30}{En cours} | \colorbox{red!30}{En retard}

\subsection{Critères d'Escalade}

\warningbox{
\textbf{DÉCLENCHEURS D'ESCALADE IMMÉDIATE:}

Définissez clairement les conditions qui nécessitent une escalade:
\begin{enumerate}
\item [Condition 1 - Ex: Dépassement budgétaire > 10\%]
\item [Condition 2 - Ex: Retard sur jalons critiques > 2 semaines]
\item [Condition 3 - Ex: Problème de sécurité identifié]
\item [Condition 4 - Ex: Résistance organisationnelle majeure]
\item [Condition 5 - Ex: Défaillance technique critique]
\end{enumerate}

\textbf{Processus d'escalade:} [Décrivez qui contacter, dans quel délai, et avec quelle information]
} in your main document

\section{Analyse des Risques}

\subsection{Matrice d'Évaluation des Risques}

\begin{table}[H]
\centering
\caption{Évaluation des Risques Principaux}
\label{tab:risk-matrix}
\begin{tabular}{|p{0.25\textwidth}|c|c|c|p{0.3\textwidth}|}
\hline
\rowcolor{DollaramaGreen!30}
\textbf{\color{white}Risque} & 
\textbf{\color{white}Probabilité} & 
\textbf{\color{white}Impact} & 
\textbf{\color{white}Score} & 
\textbf{\color{white}Mitigation} \\
\hline
[Risque technique] & \cellcolor{yellow!50}\textbf{M} & \cellcolor{yellow!50}\textbf{M} & \cellcolor{yellow!50}\textbf{6} & [Stratégie de mitigation] \\
\hline
\rowcolor{gray!10}
[Risque financier] & \cellcolor{green!30}\textbf{F} & \cellcolor{red!30}\textbf{É} & \cellcolor{yellow!50}\textbf{6} & [Stratégie de mitigation] \\
\hline
[Risque opérationnel] & \cellcolor{green!30}\textbf{F} & \cellcolor{yellow!50}\textbf{M} & \cellcolor{green!30}\textbf{3} & [Stratégie de mitigation] \\
\hline
\rowcolor{gray!10}
[Risque de délai] & \cellcolor{yellow!50}\textbf{M} & \cellcolor{yellow!50}\textbf{M} & \cellcolor{yellow!50}\textbf{6} & [Stratégie de mitigation] \\
\hline
[Risque organisationnel] & \cellcolor{green!30}\textbf{F} & \cellcolor{red!30}\textbf{É} & \cellcolor{yellow!50}\textbf{6} & [Stratégie de mitigation] \\
\hline
[Risque externe] & \cellcolor{green!30}\textbf{F} & \cellcolor{yellow!50}\textbf{M} & \cellcolor{green!30}\textbf{3} & [Stratégie de mitigation] \\
\hline
\end{tabular}
\end{table}

\textbf{Légende:} F = Faible, M = Moyen, É = Élevé | Score = Probabilité × Impact (1-3 scale)

% Risk Heat Map
\begin{table}[H]
\centering
\caption{Matrice de Chaleur des Risques}
\label{tab:risk-heatmap}
\begin{tabular}{|l|c|c|c|c|}
\hline
\rowcolor{DollaramaGreen!30}
\textbf{\color{white}Impact/Probabilité} & 
\textbf{\color{white}Très Faible} & 
\textbf{\color{white}Faible} & 
\textbf{\color{white}Moyenne} & 
\textbf{\color{white}Élevée} \\
\hline
\textbf{Critique} & \cellcolor{yellow!50}4 & \cellcolor{orange!50}8 & \cellcolor{red!50}12 & \cellcolor{red!70}16 \\
\hline
\textbf{Majeur} & \cellcolor{green!50}3 & \cellcolor{yellow!50}6 & \cellcolor{orange!50}9 & \cellcolor{red!50}12 \\
\hline
\textbf{Modéré} & \cellcolor{green!30}2 & \cellcolor{green!50}4 & \cellcolor{yellow!50}6 & \cellcolor{orange!50}8 \\
\hline
\textbf{Mineur} & \cellcolor{green!20}1 & \cellcolor{green!30}2 & \cellcolor{green!50}3 & \cellcolor{yellow!50}4 \\
\hline
\end{tabular}
\end{table}

\subsection{Analyse Détaillée des Risques}

\subsubsection{Risques Techniques}

\preventionbox{
\textbf{STRATÉGIES DE PRÉVENTION - RISQUES TECHNIQUES}

Cette section doit détailler les mesures préventives pour chaque risque technique identifié:
\begin{itemize}
\item \textbf{[Risque 1]:} Description de la stratégie de prévention
\item \textbf{[Risque 2]:} Description de la stratégie de prévention
\item \textbf{[Risque 3]:} Description de la stratégie de prévention
\end{itemize}

Incluez des actions spécifiques, des responsables et des échéances pour chaque mesure.
}

\subsubsection{Risques Opérationnels}

\contingencybox{
\textbf{PLANS DE CONTINGENCE - RISQUES OPÉRATIONNELS}

Décrivez les plans alternatifs si les risques se matérialisent:
\begin{itemize}
\item \textbf{[Plan A]:} Actions à prendre si [condition]
\item \textbf{[Plan B]:} Alternative si le Plan A échoue
\item \textbf{[Ressources de secours]:} Équipes/budget disponibles en urgence
\end{itemize}

Assurez-vous que chaque plan a un déclencheur clair et des responsables désignés.
}

\subsubsection{Risques Financiers}

\recoverybox{
\textbf{STRATÉGIES DE RÉCUPÉRATION - RISQUES FINANCIERS}

Plans pour minimiser l'impact financier si les risques se réalisent:
\begin{itemize}
\item \textbf{[Dépassement budget]:} Actions correctives et sources de financement alternatives
\item \textbf{[Retard ROI]:} Mesures d'accélération des bénéfices
\item \textbf{[Coûts cachés]:} Processus d'identification et de contrôle
\end{itemize}

Quantifiez l'impact potentiel et les coûts de récupération.
}

\subsection{Tableau de Bord des Risques}

\vspace{1em}
\begin{center}
\fcolorbox{DollaramaGreen}{DollaramaGreen!5}{%
\begin{minipage}{0.9\textwidth}
\vspace{0.5em}
\begin{center}
{\Large\bfseries\color{DollaramaGreen} TABLEAU DE BORD DES RISQUES}
\end{center}
\vspace{0.5em}

\begin{tabular}{p{0.22\textwidth}p{0.22\textwidth}p{0.22\textwidth}p{0.22\textwidth}}
\centering
\fcolorbox{green}{white}{%
\begin{minipage}{0.2\textwidth}
\centering
\textbf{Risques Faibles}\\
\vspace{0.2em}
{\Large\color{green}[X]}
\end{minipage}
} & 
\centering
\fcolorbox{orange}{white}{%
\begin{minipage}{0.2\textwidth}
\centering
\textbf{Risques Moyens}\\
\vspace{0.2em}
{\Large\color{orange}[Y]}
\end{minipage}
} &
\centering
\fcolorbox{red}{white}{%
\begin{minipage}{0.2\textwidth}
\centering
\textbf{Risques Élevés}\\
\vspace{0.2em}
{\Large\color{red}[Z]}
\end{minipage}
} &
\centering
\fcolorbox{DollaramaGreen}{white}{%
\begin{minipage}{0.2\textwidth}
\centering
\textbf{Mitigations Actives}\\
\vspace{0.2em}
{\Large\color{DollaramaGreen}[N]}
\end{minipage}
}
\end{tabular}

\vspace{0.5em}
\end{minipage}
}
\end{center}
\vspace{1em}

\subsection{Plan de Gestion des Risques}

\begin{longtable}{|p{0.2\textwidth}|p{0.3\textwidth}|p{0.15\textwidth}|p{0.15\textwidth}|p{0.15\textwidth}|}
\caption{Plan de Suivi des Risques} \label{tab:risk-management} \\
\hline
\rowcolor{DollaramaGreen!30}
\textbf{\color{white}Risque} & 
\textbf{\color{white}Action de Mitigation} & 
\textbf{\color{white}Responsable} & 
\textbf{\color{white}Échéance} & 
\textbf{\color{white}Statut} \\
\hline
\endfirsthead

\multicolumn{5}{c}%
{{\tablename\ \thetable{} -- Suite de la page précédente}} \\
\hline
\rowcolor{DollaramaGreen!30}
\textbf{\color{white}Risque} & 
\textbf{\color{white}Action de Mitigation} & 
\textbf{\color{white}Responsable} & 
\textbf{\color{white}Échéance} & 
\textbf{\color{white}Statut} \\
\hline
\endhead

\hline \multicolumn{5}{|r|}{{Suite sur la page suivante}} \\ \hline
\endfoot

\hline
\endlastfoot

[Risque 1] & [Action détaillée] & [Nom] & [Date] & \cellcolor{yellow!30}[Statut] \\
\hline
\rowcolor{gray!10}
[Risque 2] & [Action détaillée] & [Nom] & [Date] & \cellcolor{green!30}[Statut] \\
\hline
[Risque 3] & [Action détaillée] & [Nom] & [Date] & \cellcolor{green!30}[Statut] \\
\hline
\rowcolor{gray!10}
[Risque 4] & [Action détaillée] & [Nom] & [Date] & \cellcolor{yellow!30}[Statut] \\
\hline
[Risque 5] & [Action détaillée] & [Nom] & [Date] & \cellcolor{red!30}[Statut] \\
\hline
\rowcolor{gray!10}
[Risque 6] & [Action détaillée] & [Nom] & [Date] & \cellcolor{green!30}[Statut] \\
\hline
\end{longtable}

\textbf{Légende Statuts:} \colorbox{green!30}{Planifié} | \colorbox{yellow!30}{En cours} | \colorbox{red!30}{En retard}

\subsection{Critères d'Escalade}

\warningbox{
\textbf{DÉCLENCHEURS D'ESCALADE IMMÉDIATE:}

Définissez clairement les conditions qui nécessitent une escalade:
\begin{enumerate}
\item [Condition 1 - Ex: Dépassement budgétaire > 10\%]
\item [Condition 2 - Ex: Retard sur jalons critiques > 2 semaines]
\item [Condition 3 - Ex: Problème de sécurité identifié]
\item [Condition 4 - Ex: Résistance organisationnelle majeure]
\item [Condition 5 - Ex: Défaillance technique critique]
\end{enumerate}

\textbf{Processus d'escalade:} [Décrivez qui contacter, dans quel délai, et avec quelle information]
} in your main document

\section{Analyse des Risques}

\subsection{Matrice d'Évaluation des Risques}

\begin{table}[H]
\centering
\caption{Évaluation des Risques Principaux}
\label{tab:risk-matrix}
\begin{tabular}{|p{0.25\textwidth}|c|c|c|p{0.3\textwidth}|}
\hline
\rowcolor{DollaramaGreen!30}
\textbf{\color{white}Risque} & 
\textbf{\color{white}Probabilité} & 
\textbf{\color{white}Impact} & 
\textbf{\color{white}Score} & 
\textbf{\color{white}Mitigation} \\
\hline
[Risque technique] & \cellcolor{yellow!50}\textbf{M} & \cellcolor{yellow!50}\textbf{M} & \cellcolor{yellow!50}\textbf{6} & [Stratégie de mitigation] \\
\hline
\rowcolor{gray!10}
[Risque financier] & \cellcolor{green!30}\textbf{F} & \cellcolor{red!30}\textbf{É} & \cellcolor{yellow!50}\textbf{6} & [Stratégie de mitigation] \\
\hline
[Risque opérationnel] & \cellcolor{green!30}\textbf{F} & \cellcolor{yellow!50}\textbf{M} & \cellcolor{green!30}\textbf{3} & [Stratégie de mitigation] \\
\hline
\rowcolor{gray!10}
[Risque de délai] & \cellcolor{yellow!50}\textbf{M} & \cellcolor{yellow!50}\textbf{M} & \cellcolor{yellow!50}\textbf{6} & [Stratégie de mitigation] \\
\hline
[Risque organisationnel] & \cellcolor{green!30}\textbf{F} & \cellcolor{red!30}\textbf{É} & \cellcolor{yellow!50}\textbf{6} & [Stratégie de mitigation] \\
\hline
[Risque externe] & \cellcolor{green!30}\textbf{F} & \cellcolor{yellow!50}\textbf{M} & \cellcolor{green!30}\textbf{3} & [Stratégie de mitigation] \\
\hline
\end{tabular}
\end{table}

\textbf{Légende:} F = Faible, M = Moyen, É = Élevé | Score = Probabilité × Impact (1-3 scale)

% Risk Heat Map
\begin{table}[H]
\centering
\caption{Matrice de Chaleur des Risques}
\label{tab:risk-heatmap}
\begin{tabular}{|l|c|c|c|c|}
\hline
\rowcolor{DollaramaGreen!30}
\textbf{\color{white}Impact/Probabilité} & 
\textbf{\color{white}Très Faible} & 
\textbf{\color{white}Faible} & 
\textbf{\color{white}Moyenne} & 
\textbf{\color{white}Élevée} \\
\hline
\textbf{Critique} & \cellcolor{yellow!50}4 & \cellcolor{orange!50}8 & \cellcolor{red!50}12 & \cellcolor{red!70}16 \\
\hline
\textbf{Majeur} & \cellcolor{green!50}3 & \cellcolor{yellow!50}6 & \cellcolor{orange!50}9 & \cellcolor{red!50}12 \\
\hline
\textbf{Modéré} & \cellcolor{green!30}2 & \cellcolor{green!50}4 & \cellcolor{yellow!50}6 & \cellcolor{orange!50}8 \\
\hline
\textbf{Mineur} & \cellcolor{green!20}1 & \cellcolor{green!30}2 & \cellcolor{green!50}3 & \cellcolor{yellow!50}4 \\
\hline
\end{tabular}
\end{table}

\subsection{Analyse Détaillée des Risques}

\subsubsection{Risques Techniques}

\preventionbox{
\textbf{STRATÉGIES DE PRÉVENTION - RISQUES TECHNIQUES}

Cette section doit détailler les mesures préventives pour chaque risque technique identifié:
\begin{itemize}
\item \textbf{[Risque 1]:} Description de la stratégie de prévention
\item \textbf{[Risque 2]:} Description de la stratégie de prévention
\item \textbf{[Risque 3]:} Description de la stratégie de prévention
\end{itemize}

Incluez des actions spécifiques, des responsables et des échéances pour chaque mesure.
}

\subsubsection{Risques Opérationnels}

\contingencybox{
\textbf{PLANS DE CONTINGENCE - RISQUES OPÉRATIONNELS}

Décrivez les plans alternatifs si les risques se matérialisent:
\begin{itemize}
\item \textbf{[Plan A]:} Actions à prendre si [condition]
\item \textbf{[Plan B]:} Alternative si le Plan A échoue
\item \textbf{[Ressources de secours]:} Équipes/budget disponibles en urgence
\end{itemize}

Assurez-vous que chaque plan a un déclencheur clair et des responsables désignés.
}

\subsubsection{Risques Financiers}

\recoverybox{
\textbf{STRATÉGIES DE RÉCUPÉRATION - RISQUES FINANCIERS}

Plans pour minimiser l'impact financier si les risques se réalisent:
\begin{itemize}
\item \textbf{[Dépassement budget]:} Actions correctives et sources de financement alternatives
\item \textbf{[Retard ROI]:} Mesures d'accélération des bénéfices
\item \textbf{[Coûts cachés]:} Processus d'identification et de contrôle
\end{itemize}

Quantifiez l'impact potentiel et les coûts de récupération.
}

\subsection{Tableau de Bord des Risques}

\vspace{1em}
\begin{center}
\fcolorbox{DollaramaGreen}{DollaramaGreen!5}{%
\begin{minipage}{0.9\textwidth}
\vspace{0.5em}
\begin{center}
{\Large\bfseries\color{DollaramaGreen} TABLEAU DE BORD DES RISQUES}
\end{center}
\vspace{0.5em}

\begin{tabular}{p{0.22\textwidth}p{0.22\textwidth}p{0.22\textwidth}p{0.22\textwidth}}
\centering
\fcolorbox{green}{white}{%
\begin{minipage}{0.2\textwidth}
\centering
\textbf{Risques Faibles}\\
\vspace{0.2em}
{\Large\color{green}[X]}
\end{minipage}
} & 
\centering
\fcolorbox{orange}{white}{%
\begin{minipage}{0.2\textwidth}
\centering
\textbf{Risques Moyens}\\
\vspace{0.2em}
{\Large\color{orange}[Y]}
\end{minipage}
} &
\centering
\fcolorbox{red}{white}{%
\begin{minipage}{0.2\textwidth}
\centering
\textbf{Risques Élevés}\\
\vspace{0.2em}
{\Large\color{red}[Z]}
\end{minipage}
} &
\centering
\fcolorbox{DollaramaGreen}{white}{%
\begin{minipage}{0.2\textwidth}
\centering
\textbf{Mitigations Actives}\\
\vspace{0.2em}
{\Large\color{DollaramaGreen}[N]}
\end{minipage}
}
\end{tabular}

\vspace{0.5em}
\end{minipage}
}
\end{center}
\vspace{1em}

\subsection{Plan de Gestion des Risques}

\begin{longtable}{|p{0.2\textwidth}|p{0.3\textwidth}|p{0.15\textwidth}|p{0.15\textwidth}|p{0.15\textwidth}|}
\caption{Plan de Suivi des Risques} \label{tab:risk-management} \\
\hline
\rowcolor{DollaramaGreen!30}
\textbf{\color{white}Risque} & 
\textbf{\color{white}Action de Mitigation} & 
\textbf{\color{white}Responsable} & 
\textbf{\color{white}Échéance} & 
\textbf{\color{white}Statut} \\
\hline
\endfirsthead

\multicolumn{5}{c}%
{{\tablename\ \thetable{} -- Suite de la page précédente}} \\
\hline
\rowcolor{DollaramaGreen!30}
\textbf{\color{white}Risque} & 
\textbf{\color{white}Action de Mitigation} & 
\textbf{\color{white}Responsable} & 
\textbf{\color{white}Échéance} & 
\textbf{\color{white}Statut} \\
\hline
\endhead

\hline \multicolumn{5}{|r|}{{Suite sur la page suivante}} \\ \hline
\endfoot

\hline
\endlastfoot

[Risque 1] & [Action détaillée] & [Nom] & [Date] & \cellcolor{yellow!30}[Statut] \\
\hline
\rowcolor{gray!10}
[Risque 2] & [Action détaillée] & [Nom] & [Date] & \cellcolor{green!30}[Statut] \\
\hline
[Risque 3] & [Action détaillée] & [Nom] & [Date] & \cellcolor{green!30}[Statut] \\
\hline
\rowcolor{gray!10}
[Risque 4] & [Action détaillée] & [Nom] & [Date] & \cellcolor{yellow!30}[Statut] \\
\hline
[Risque 5] & [Action détaillée] & [Nom] & [Date] & \cellcolor{red!30}[Statut] \\
\hline
\rowcolor{gray!10}
[Risque 6] & [Action détaillée] & [Nom] & [Date] & \cellcolor{green!30}[Statut] \\
\hline
\end{longtable}

\textbf{Légende Statuts:} \colorbox{green!30}{Planifié} | \colorbox{yellow!30}{En cours} | \colorbox{red!30}{En retard}

\subsection{Critères d'Escalade}

\warningbox{
\textbf{DÉCLENCHEURS D'ESCALADE IMMÉDIATE:}

Définissez clairement les conditions qui nécessitent une escalade:
\begin{enumerate}
\item [Condition 1 - Ex: Dépassement budgétaire > 10\%]
\item [Condition 2 - Ex: Retard sur jalons critiques > 2 semaines]
\item [Condition 3 - Ex: Problème de sécurité identifié]
\item [Condition 4 - Ex: Résistance organisationnelle majeure]
\item [Condition 5 - Ex: Défaillance technique critique]
\end{enumerate}

\textbf{Processus d'escalade:} [Décrivez qui contacter, dans quel délai, et avec quelle information]
}