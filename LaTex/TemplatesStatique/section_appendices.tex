% Modular Appendices Section
% This file contains ONLY the appendices content
% Usage: % Modular Appendices Section
% This file contains ONLY the appendices content
% Usage: % Modular Appendices Section
% This file contains ONLY the appendices content
% Usage: % Modular Appendices Section
% This file contains ONLY the appendices content
% Usage: \input{section_appendices} in your main document

\clearpage
\appendix

% Reset page numbering for appendices
\pagenumbering{Roman}
\setcounter{page}{1}

\section{Spécifications Techniques Détaillées}
\label{app:technical-specs}

\subsection{[Composant Principal] - Spécifications}

\begin{table}[H]
\centering
\caption{Spécifications Techniques Complètes}
\label{tab:technical-specs}
\begin{tabular}{|p{0.3\textwidth}|p{0.6\textwidth}|}
\hline
\rowcolor{DollaramaGreen!30}
\textbf{\color{white}Composant} & \textbf{\color{white}Spécification Détaillée} \\
\hline
[Élément technique 1] & [Description détaillée, version, capacité] \\
\hline
\rowcolor{gray!10}
[Élément technique 2] & [Description détaillée, version, capacité] \\
\hline
[Élément technique 3] & [Description détaillée, version, capacité] \\
\hline
\rowcolor{gray!10}
[Élément technique 4] & [Description détaillée, version, capacité] \\
\hline
[Élément technique 5] & [Description détaillée, version, capacité] \\
\hline
\rowcolor{gray!10}
[Élément technique 6] & [Description détaillée, version, capacité] \\
\hline
[Certification/Conformité] & [Standards et certifications requis] \\
\hline
\end{tabular}
\end{table}

\subsection{Configuration Logicielle}

\begin{table}[H]
\centering
\caption{Stack Logiciel et Versions}
\label{tab:software-stack}
\begin{tabular}{|p{0.25\textwidth}|p{0.25\textwidth}|p{0.4\textwidth}|}
\hline
\rowcolor{DollaramaGreen!30}
\textbf{\color{white}Couche} & \textbf{\color{white}Composant} & \textbf{\color{white}Version/Configuration} \\
\hline
[Système d'exploitation] & [Nom] & [Version et configuration] \\
\hline
\rowcolor{gray!10}
[Base de données] & [Nom] & [Version et configuration] \\
\hline
[Serveur application] & [Nom] & [Version et configuration] \\
\hline
\rowcolor{gray!10}
[Framework] & [Nom] & [Version et configuration] \\
\hline
[Sécurité] & [Nom] & [Version et configuration] \\
\hline
\rowcolor{gray!10}
[Monitoring] & [Nom] & [Version et configuration] \\
\hline
[Backup] & [Nom] & [Version et configuration] \\
\hline
\end{tabular}
\end{table}

\section{Architecture Détaillée}
\label{app:architecture}

\subsection{Diagramme d'Architecture}

\begin{figure}[H]
\centering
\begin{tikzpicture}[node distance=3cm, auto, scale=0.8, transform shape]
    % Exemple de diagramme d'architecture
    % Remplacez par votre propre diagramme
    
    % Composants réseau
    \node [draw, rectangle, fill=DollaramaGreen!20, text width=2cm, text centered] (component1) {Composant 1};
    \node [draw, rectangle, below of=component1, fill=blue!20, text width=2cm, text centered] (component2) {Composant 2};
    \node [draw, rectangle, right of=component1, fill=DollaramaYellow!20, text width=2cm, text centered] (component3) {Composant 3};
    \node [draw, rectangle, below of=component3, fill=gray!20, text width=2cm, text centered] (component4) {Composant 4};
    
    % Connexions
    \draw [->] (component1) -- (component2);
    \draw [->] (component1) -- (component3);
    \draw [->] (component3) -- (component4);
    \draw [<->] (component2) -- (component4);
    
    % Annotations
    \node [text width=3cm] at ([xshift=4cm]component3.east) {Zone sécurisée};
\end{tikzpicture}
\caption{Architecture Système Proposée}
\label{fig:architecture}
\end{figure}

\textbf{Description:} [Expliquez les composants clés et les flux de données]

\section{Analyse Détaillée des Coûts}
\label{app:detailed-costs}

\subsection{Modèle de Coûts Complet}

\begin{longtable}{|p{0.3\textwidth}|p{0.15\textwidth}|p{0.15\textwidth}|p{0.15\textwidth}|p{0.2\textwidth}|}
\caption{Ventilation Détaillée des Coûts} \label{tab:detailed-costs} \\
\hline
\rowcolor{DollaramaGreen!30}
\textbf{\color{white}Catégorie} & 
\textbf{\color{white}Année 1} & 
\textbf{\color{white}Année 2} & 
\textbf{\color{white}Année 3} & 
\textbf{\color{white}Total 5 ans} \\
\hline
\endfirsthead

\multicolumn{5}{c}%
{{\tablename\ \thetable{} -- Suite de la page précédente}} \\
\hline
\rowcolor{DollaramaGreen!30}
\textbf{\color{white}Catégorie} & 
\textbf{\color{white}Année 1} & 
\textbf{\color{white}Année 2} & 
\textbf{\color{white}Année 3} & 
\textbf{\color{white}Total 5 ans} \\
\hline
\endhead

\hline \multicolumn{5}{|r|}{{Suite sur la page suivante}} \\ \hline
\endfoot

\hline
\endlastfoot

\textbf{COÛTS CAPEX} & & & & \\
\hline
[Matériel] & \$[Montant] & \$[Montant] & \$[Montant] & \$[Total] \\
\hline
\rowcolor{gray!10}
[Logiciels/Licences] & \$[Montant] & \$[Montant] & \$[Montant] & \$[Total] \\
\hline
[Infrastructure] & \$[Montant] & \$[Montant] & \$[Montant] & \$[Total] \\
\hline
\textbf{COÛTS OPEX} & & & & \\
\hline
[Support/Maintenance] & \$[Montant] & \$[Montant] & \$[Montant] & \$[Total] \\
\hline
\rowcolor{gray!10}
[Personnel] & \$[Montant] & \$[Montant] & \$[Montant] & \$[Total] \\
\hline
[Formation continue] & \$[Montant] & \$[Montant] & \$[Montant] & \$[Total] \\
\hline
\rowcolor{DollaramaGreen!20}
\textbf{TOTAL} & \textbf{\$[Total]} & \textbf{\$[Total]} & \textbf{\$[Total]} & \textbf{\$[TOTAL]} \\
\hline
\end{longtable}

\section{Documentation des Procédures}
\label{app:procedures}

\subsection{Procédure de [Nom du Processus]}

\begin{enumerate}
\item \textbf{[Étape 1 - Titre]}
    \begin{itemize}
    \item [Action spécifique à effectuer]
    \item [Vérification ou contrôle requis]
    \item [Personne responsable]
    \item [Temps estimé]
    \end{itemize}

\item \textbf{[Étape 2 - Titre]}
    \begin{itemize}
    \item [Action spécifique à effectuer]
    \item [Vérification ou contrôle requis]
    \item [Personne responsable]
    \item [Temps estimé]
    \end{itemize}

\item \textbf{[Étape 3 - Titre]}
    \begin{itemize}
    \item [Action spécifique à effectuer]
    \item [Vérification ou contrôle requis]
    \item [Personne responsable]
    \item [Temps estimé]
    \end{itemize}

\item \textbf{[Étape finale - Validation]}
    \begin{itemize}
    \item [Critères de validation]
    \item [Documentation à produire]
    \item [Approbations requises]
    \item [Archivage et communication]
    \end{itemize}
\end{enumerate}

\subsection{Guide d'Utilisation Rapide}

\begin{table}[H]
\centering
\caption{Procédure Simplifiée - Aide-Mémoire}
\label{tab:quick-guide}
\begin{tabular}{|p{0.3\textwidth}|p{0.65\textwidth}|}
\hline
\rowcolor{DollaramaGreen!30}
\textbf{\color{white}Étape} & \textbf{\color{white}Action} \\
\hline
1. [Titre court] & [Description de l'action en une phrase] \\
\hline
\rowcolor{gray!10}
2. [Titre court] & [Description de l'action en une phrase] \\
\hline
3. [Titre court] & [Description de l'action en une phrase] \\
\hline
\rowcolor{gray!10}
4. [Titre court] & [Description de l'action en une phrase] \\
\hline
5. [Titre court] & [Description de l'action en une phrase] \\
\hline
\rowcolor{gray!10}
6. [Titre court] & [Description de l'action en une phrase] \\
\hline
7. [Titre court] & [Description de l'action en une phrase] \\
\hline
\end{tabular}
\end{table}

\section{Standards et Références}
\label{app:standards}

\subsection{Standards Techniques Applicables}

\begin{itemize}
\item \textbf{[Standard 1]} - [Description et applicabilité au projet]
\item \textbf{[Standard 2]} - [Description et applicabilité au projet]
\item \textbf{[Standard 3]} - [Description et applicabilité au projet]
\item \textbf{[Standard 4]} - [Description et applicabilité au projet]
\item \textbf{[Standard 5]} - [Description et applicabilité au projet]
\end{itemize}

\subsection{Références Réglementaires}

\begin{itemize}
\item \textbf{[Réglementation 1]} - [Impact sur le projet]
\item \textbf{[Réglementation 2]} - [Impact sur le projet]
\item \textbf{[Réglementation 3]} - [Impact sur le projet]
\end{itemize}

\subsection{Documentation de Référence}

\begin{enumerate}
\item [Organisation] ([Année]). \textit{[Titre du document]}
\item [Organisation] ([Année]). \textit{[Titre du document]}
\item [Organisation] ([Année]). \textit{[Titre du document]}
\item [Organisation] ([Année]). \textit{[Titre du document]}
\item [Organisation] ([Année]). \textit{[Titre du document]}
\end{enumerate}

\section{Glossaire}
\label{app:glossary}

\begin{longtable}{|p{0.25\textwidth}|p{0.7\textwidth}|}
\caption{Glossaire des Termes Techniques} \label{tab:glossary} \\
\hline
\rowcolor{DollaramaGreen!30}
\textbf{\color{white}Terme} & \textbf{\color{white}Définition} \\
\hline
\endfirsthead

\multicolumn{2}{c}%
{{\tablename\ \thetable{} -- Suite de la page précédente}} \\
\hline
\rowcolor{DollaramaGreen!30}
\textbf{\color{white}Terme} & \textbf{\color{white}Définition} \\
\hline
\endhead

\hline \multicolumn{2}{|r|}{{Suite sur la page suivante}} \\ \hline
\endfoot

\hline
\endlastfoot

[Acronyme/Terme 1] & [Définition complète et contexte d'utilisation] \\
\hline
\rowcolor{gray!10}
[Acronyme/Terme 2] & [Définition complète et contexte d'utilisation] \\
\hline
[Acronyme/Terme 3] & [Définition complète et contexte d'utilisation] \\
\hline
\rowcolor{gray!10}
[Acronyme/Terme 4] & [Définition complète et contexte d'utilisation] \\
\hline
[Acronyme/Terme 5] & [Définition complète et contexte d'utilisation] \\
\hline
\end{longtable} in your main document

\clearpage
\appendix

% Reset page numbering for appendices
\pagenumbering{Roman}
\setcounter{page}{1}

\section{Spécifications Techniques Détaillées}
\label{app:technical-specs}

\subsection{[Composant Principal] - Spécifications}

\begin{table}[H]
\centering
\caption{Spécifications Techniques Complètes}
\label{tab:technical-specs}
\begin{tabular}{|p{0.3\textwidth}|p{0.6\textwidth}|}
\hline
\rowcolor{DollaramaGreen!30}
\textbf{\color{white}Composant} & \textbf{\color{white}Spécification Détaillée} \\
\hline
[Élément technique 1] & [Description détaillée, version, capacité] \\
\hline
\rowcolor{gray!10}
[Élément technique 2] & [Description détaillée, version, capacité] \\
\hline
[Élément technique 3] & [Description détaillée, version, capacité] \\
\hline
\rowcolor{gray!10}
[Élément technique 4] & [Description détaillée, version, capacité] \\
\hline
[Élément technique 5] & [Description détaillée, version, capacité] \\
\hline
\rowcolor{gray!10}
[Élément technique 6] & [Description détaillée, version, capacité] \\
\hline
[Certification/Conformité] & [Standards et certifications requis] \\
\hline
\end{tabular}
\end{table}

\subsection{Configuration Logicielle}

\begin{table}[H]
\centering
\caption{Stack Logiciel et Versions}
\label{tab:software-stack}
\begin{tabular}{|p{0.25\textwidth}|p{0.25\textwidth}|p{0.4\textwidth}|}
\hline
\rowcolor{DollaramaGreen!30}
\textbf{\color{white}Couche} & \textbf{\color{white}Composant} & \textbf{\color{white}Version/Configuration} \\
\hline
[Système d'exploitation] & [Nom] & [Version et configuration] \\
\hline
\rowcolor{gray!10}
[Base de données] & [Nom] & [Version et configuration] \\
\hline
[Serveur application] & [Nom] & [Version et configuration] \\
\hline
\rowcolor{gray!10}
[Framework] & [Nom] & [Version et configuration] \\
\hline
[Sécurité] & [Nom] & [Version et configuration] \\
\hline
\rowcolor{gray!10}
[Monitoring] & [Nom] & [Version et configuration] \\
\hline
[Backup] & [Nom] & [Version et configuration] \\
\hline
\end{tabular}
\end{table}

\section{Architecture Détaillée}
\label{app:architecture}

\subsection{Diagramme d'Architecture}

\begin{figure}[H]
\centering
\begin{tikzpicture}[node distance=3cm, auto, scale=0.8, transform shape]
    % Exemple de diagramme d'architecture
    % Remplacez par votre propre diagramme
    
    % Composants réseau
    \node [draw, rectangle, fill=DollaramaGreen!20, text width=2cm, text centered] (component1) {Composant 1};
    \node [draw, rectangle, below of=component1, fill=blue!20, text width=2cm, text centered] (component2) {Composant 2};
    \node [draw, rectangle, right of=component1, fill=DollaramaYellow!20, text width=2cm, text centered] (component3) {Composant 3};
    \node [draw, rectangle, below of=component3, fill=gray!20, text width=2cm, text centered] (component4) {Composant 4};
    
    % Connexions
    \draw [->] (component1) -- (component2);
    \draw [->] (component1) -- (component3);
    \draw [->] (component3) -- (component4);
    \draw [<->] (component2) -- (component4);
    
    % Annotations
    \node [text width=3cm] at ([xshift=4cm]component3.east) {Zone sécurisée};
\end{tikzpicture}
\caption{Architecture Système Proposée}
\label{fig:architecture}
\end{figure}

\textbf{Description:} [Expliquez les composants clés et les flux de données]

\section{Analyse Détaillée des Coûts}
\label{app:detailed-costs}

\subsection{Modèle de Coûts Complet}

\begin{longtable}{|p{0.3\textwidth}|p{0.15\textwidth}|p{0.15\textwidth}|p{0.15\textwidth}|p{0.2\textwidth}|}
\caption{Ventilation Détaillée des Coûts} \label{tab:detailed-costs} \\
\hline
\rowcolor{DollaramaGreen!30}
\textbf{\color{white}Catégorie} & 
\textbf{\color{white}Année 1} & 
\textbf{\color{white}Année 2} & 
\textbf{\color{white}Année 3} & 
\textbf{\color{white}Total 5 ans} \\
\hline
\endfirsthead

\multicolumn{5}{c}%
{{\tablename\ \thetable{} -- Suite de la page précédente}} \\
\hline
\rowcolor{DollaramaGreen!30}
\textbf{\color{white}Catégorie} & 
\textbf{\color{white}Année 1} & 
\textbf{\color{white}Année 2} & 
\textbf{\color{white}Année 3} & 
\textbf{\color{white}Total 5 ans} \\
\hline
\endhead

\hline \multicolumn{5}{|r|}{{Suite sur la page suivante}} \\ \hline
\endfoot

\hline
\endlastfoot

\textbf{COÛTS CAPEX} & & & & \\
\hline
[Matériel] & \$[Montant] & \$[Montant] & \$[Montant] & \$[Total] \\
\hline
\rowcolor{gray!10}
[Logiciels/Licences] & \$[Montant] & \$[Montant] & \$[Montant] & \$[Total] \\
\hline
[Infrastructure] & \$[Montant] & \$[Montant] & \$[Montant] & \$[Total] \\
\hline
\textbf{COÛTS OPEX} & & & & \\
\hline
[Support/Maintenance] & \$[Montant] & \$[Montant] & \$[Montant] & \$[Total] \\
\hline
\rowcolor{gray!10}
[Personnel] & \$[Montant] & \$[Montant] & \$[Montant] & \$[Total] \\
\hline
[Formation continue] & \$[Montant] & \$[Montant] & \$[Montant] & \$[Total] \\
\hline
\rowcolor{DollaramaGreen!20}
\textbf{TOTAL} & \textbf{\$[Total]} & \textbf{\$[Total]} & \textbf{\$[Total]} & \textbf{\$[TOTAL]} \\
\hline
\end{longtable}

\section{Documentation des Procédures}
\label{app:procedures}

\subsection{Procédure de [Nom du Processus]}

\begin{enumerate}
\item \textbf{[Étape 1 - Titre]}
    \begin{itemize}
    \item [Action spécifique à effectuer]
    \item [Vérification ou contrôle requis]
    \item [Personne responsable]
    \item [Temps estimé]
    \end{itemize}

\item \textbf{[Étape 2 - Titre]}
    \begin{itemize}
    \item [Action spécifique à effectuer]
    \item [Vérification ou contrôle requis]
    \item [Personne responsable]
    \item [Temps estimé]
    \end{itemize}

\item \textbf{[Étape 3 - Titre]}
    \begin{itemize}
    \item [Action spécifique à effectuer]
    \item [Vérification ou contrôle requis]
    \item [Personne responsable]
    \item [Temps estimé]
    \end{itemize}

\item \textbf{[Étape finale - Validation]}
    \begin{itemize}
    \item [Critères de validation]
    \item [Documentation à produire]
    \item [Approbations requises]
    \item [Archivage et communication]
    \end{itemize}
\end{enumerate}

\subsection{Guide d'Utilisation Rapide}

\begin{table}[H]
\centering
\caption{Procédure Simplifiée - Aide-Mémoire}
\label{tab:quick-guide}
\begin{tabular}{|p{0.3\textwidth}|p{0.65\textwidth}|}
\hline
\rowcolor{DollaramaGreen!30}
\textbf{\color{white}Étape} & \textbf{\color{white}Action} \\
\hline
1. [Titre court] & [Description de l'action en une phrase] \\
\hline
\rowcolor{gray!10}
2. [Titre court] & [Description de l'action en une phrase] \\
\hline
3. [Titre court] & [Description de l'action en une phrase] \\
\hline
\rowcolor{gray!10}
4. [Titre court] & [Description de l'action en une phrase] \\
\hline
5. [Titre court] & [Description de l'action en une phrase] \\
\hline
\rowcolor{gray!10}
6. [Titre court] & [Description de l'action en une phrase] \\
\hline
7. [Titre court] & [Description de l'action en une phrase] \\
\hline
\end{tabular}
\end{table}

\section{Standards et Références}
\label{app:standards}

\subsection{Standards Techniques Applicables}

\begin{itemize}
\item \textbf{[Standard 1]} - [Description et applicabilité au projet]
\item \textbf{[Standard 2]} - [Description et applicabilité au projet]
\item \textbf{[Standard 3]} - [Description et applicabilité au projet]
\item \textbf{[Standard 4]} - [Description et applicabilité au projet]
\item \textbf{[Standard 5]} - [Description et applicabilité au projet]
\end{itemize}

\subsection{Références Réglementaires}

\begin{itemize}
\item \textbf{[Réglementation 1]} - [Impact sur le projet]
\item \textbf{[Réglementation 2]} - [Impact sur le projet]
\item \textbf{[Réglementation 3]} - [Impact sur le projet]
\end{itemize}

\subsection{Documentation de Référence}

\begin{enumerate}
\item [Organisation] ([Année]). \textit{[Titre du document]}
\item [Organisation] ([Année]). \textit{[Titre du document]}
\item [Organisation] ([Année]). \textit{[Titre du document]}
\item [Organisation] ([Année]). \textit{[Titre du document]}
\item [Organisation] ([Année]). \textit{[Titre du document]}
\end{enumerate}

\section{Glossaire}
\label{app:glossary}

\begin{longtable}{|p{0.25\textwidth}|p{0.7\textwidth}|}
\caption{Glossaire des Termes Techniques} \label{tab:glossary} \\
\hline
\rowcolor{DollaramaGreen!30}
\textbf{\color{white}Terme} & \textbf{\color{white}Définition} \\
\hline
\endfirsthead

\multicolumn{2}{c}%
{{\tablename\ \thetable{} -- Suite de la page précédente}} \\
\hline
\rowcolor{DollaramaGreen!30}
\textbf{\color{white}Terme} & \textbf{\color{white}Définition} \\
\hline
\endhead

\hline \multicolumn{2}{|r|}{{Suite sur la page suivante}} \\ \hline
\endfoot

\hline
\endlastfoot

[Acronyme/Terme 1] & [Définition complète et contexte d'utilisation] \\
\hline
\rowcolor{gray!10}
[Acronyme/Terme 2] & [Définition complète et contexte d'utilisation] \\
\hline
[Acronyme/Terme 3] & [Définition complète et contexte d'utilisation] \\
\hline
\rowcolor{gray!10}
[Acronyme/Terme 4] & [Définition complète et contexte d'utilisation] \\
\hline
[Acronyme/Terme 5] & [Définition complète et contexte d'utilisation] \\
\hline
\end{longtable} in your main document

\clearpage
\appendix

% Reset page numbering for appendices
\pagenumbering{Roman}
\setcounter{page}{1}

\section{Spécifications Techniques Détaillées}
\label{app:technical-specs}

\subsection{[Composant Principal] - Spécifications}

\begin{table}[H]
\centering
\caption{Spécifications Techniques Complètes}
\label{tab:technical-specs}
\begin{tabular}{|p{0.3\textwidth}|p{0.6\textwidth}|}
\hline
\rowcolor{DollaramaGreen!30}
\textbf{\color{white}Composant} & \textbf{\color{white}Spécification Détaillée} \\
\hline
[Élément technique 1] & [Description détaillée, version, capacité] \\
\hline
\rowcolor{gray!10}
[Élément technique 2] & [Description détaillée, version, capacité] \\
\hline
[Élément technique 3] & [Description détaillée, version, capacité] \\
\hline
\rowcolor{gray!10}
[Élément technique 4] & [Description détaillée, version, capacité] \\
\hline
[Élément technique 5] & [Description détaillée, version, capacité] \\
\hline
\rowcolor{gray!10}
[Élément technique 6] & [Description détaillée, version, capacité] \\
\hline
[Certification/Conformité] & [Standards et certifications requis] \\
\hline
\end{tabular}
\end{table}

\subsection{Configuration Logicielle}

\begin{table}[H]
\centering
\caption{Stack Logiciel et Versions}
\label{tab:software-stack}
\begin{tabular}{|p{0.25\textwidth}|p{0.25\textwidth}|p{0.4\textwidth}|}
\hline
\rowcolor{DollaramaGreen!30}
\textbf{\color{white}Couche} & \textbf{\color{white}Composant} & \textbf{\color{white}Version/Configuration} \\
\hline
[Système d'exploitation] & [Nom] & [Version et configuration] \\
\hline
\rowcolor{gray!10}
[Base de données] & [Nom] & [Version et configuration] \\
\hline
[Serveur application] & [Nom] & [Version et configuration] \\
\hline
\rowcolor{gray!10}
[Framework] & [Nom] & [Version et configuration] \\
\hline
[Sécurité] & [Nom] & [Version et configuration] \\
\hline
\rowcolor{gray!10}
[Monitoring] & [Nom] & [Version et configuration] \\
\hline
[Backup] & [Nom] & [Version et configuration] \\
\hline
\end{tabular}
\end{table}

\section{Architecture Détaillée}
\label{app:architecture}

\subsection{Diagramme d'Architecture}

\begin{figure}[H]
\centering
\begin{tikzpicture}[node distance=3cm, auto, scale=0.8, transform shape]
    % Exemple de diagramme d'architecture
    % Remplacez par votre propre diagramme
    
    % Composants réseau
    \node [draw, rectangle, fill=DollaramaGreen!20, text width=2cm, text centered] (component1) {Composant 1};
    \node [draw, rectangle, below of=component1, fill=blue!20, text width=2cm, text centered] (component2) {Composant 2};
    \node [draw, rectangle, right of=component1, fill=DollaramaYellow!20, text width=2cm, text centered] (component3) {Composant 3};
    \node [draw, rectangle, below of=component3, fill=gray!20, text width=2cm, text centered] (component4) {Composant 4};
    
    % Connexions
    \draw [->] (component1) -- (component2);
    \draw [->] (component1) -- (component3);
    \draw [->] (component3) -- (component4);
    \draw [<->] (component2) -- (component4);
    
    % Annotations
    \node [text width=3cm] at ([xshift=4cm]component3.east) {Zone sécurisée};
\end{tikzpicture}
\caption{Architecture Système Proposée}
\label{fig:architecture}
\end{figure}

\textbf{Description:} [Expliquez les composants clés et les flux de données]

\section{Analyse Détaillée des Coûts}
\label{app:detailed-costs}

\subsection{Modèle de Coûts Complet}

\begin{longtable}{|p{0.3\textwidth}|p{0.15\textwidth}|p{0.15\textwidth}|p{0.15\textwidth}|p{0.2\textwidth}|}
\caption{Ventilation Détaillée des Coûts} \label{tab:detailed-costs} \\
\hline
\rowcolor{DollaramaGreen!30}
\textbf{\color{white}Catégorie} & 
\textbf{\color{white}Année 1} & 
\textbf{\color{white}Année 2} & 
\textbf{\color{white}Année 3} & 
\textbf{\color{white}Total 5 ans} \\
\hline
\endfirsthead

\multicolumn{5}{c}%
{{\tablename\ \thetable{} -- Suite de la page précédente}} \\
\hline
\rowcolor{DollaramaGreen!30}
\textbf{\color{white}Catégorie} & 
\textbf{\color{white}Année 1} & 
\textbf{\color{white}Année 2} & 
\textbf{\color{white}Année 3} & 
\textbf{\color{white}Total 5 ans} \\
\hline
\endhead

\hline \multicolumn{5}{|r|}{{Suite sur la page suivante}} \\ \hline
\endfoot

\hline
\endlastfoot

\textbf{COÛTS CAPEX} & & & & \\
\hline
[Matériel] & \$[Montant] & \$[Montant] & \$[Montant] & \$[Total] \\
\hline
\rowcolor{gray!10}
[Logiciels/Licences] & \$[Montant] & \$[Montant] & \$[Montant] & \$[Total] \\
\hline
[Infrastructure] & \$[Montant] & \$[Montant] & \$[Montant] & \$[Total] \\
\hline
\textbf{COÛTS OPEX} & & & & \\
\hline
[Support/Maintenance] & \$[Montant] & \$[Montant] & \$[Montant] & \$[Total] \\
\hline
\rowcolor{gray!10}
[Personnel] & \$[Montant] & \$[Montant] & \$[Montant] & \$[Total] \\
\hline
[Formation continue] & \$[Montant] & \$[Montant] & \$[Montant] & \$[Total] \\
\hline
\rowcolor{DollaramaGreen!20}
\textbf{TOTAL} & \textbf{\$[Total]} & \textbf{\$[Total]} & \textbf{\$[Total]} & \textbf{\$[TOTAL]} \\
\hline
\end{longtable}

\section{Documentation des Procédures}
\label{app:procedures}

\subsection{Procédure de [Nom du Processus]}

\begin{enumerate}
\item \textbf{[Étape 1 - Titre]}
    \begin{itemize}
    \item [Action spécifique à effectuer]
    \item [Vérification ou contrôle requis]
    \item [Personne responsable]
    \item [Temps estimé]
    \end{itemize}

\item \textbf{[Étape 2 - Titre]}
    \begin{itemize}
    \item [Action spécifique à effectuer]
    \item [Vérification ou contrôle requis]
    \item [Personne responsable]
    \item [Temps estimé]
    \end{itemize}

\item \textbf{[Étape 3 - Titre]}
    \begin{itemize}
    \item [Action spécifique à effectuer]
    \item [Vérification ou contrôle requis]
    \item [Personne responsable]
    \item [Temps estimé]
    \end{itemize}

\item \textbf{[Étape finale - Validation]}
    \begin{itemize}
    \item [Critères de validation]
    \item [Documentation à produire]
    \item [Approbations requises]
    \item [Archivage et communication]
    \end{itemize}
\end{enumerate}

\subsection{Guide d'Utilisation Rapide}

\begin{table}[H]
\centering
\caption{Procédure Simplifiée - Aide-Mémoire}
\label{tab:quick-guide}
\begin{tabular}{|p{0.3\textwidth}|p{0.65\textwidth}|}
\hline
\rowcolor{DollaramaGreen!30}
\textbf{\color{white}Étape} & \textbf{\color{white}Action} \\
\hline
1. [Titre court] & [Description de l'action en une phrase] \\
\hline
\rowcolor{gray!10}
2. [Titre court] & [Description de l'action en une phrase] \\
\hline
3. [Titre court] & [Description de l'action en une phrase] \\
\hline
\rowcolor{gray!10}
4. [Titre court] & [Description de l'action en une phrase] \\
\hline
5. [Titre court] & [Description de l'action en une phrase] \\
\hline
\rowcolor{gray!10}
6. [Titre court] & [Description de l'action en une phrase] \\
\hline
7. [Titre court] & [Description de l'action en une phrase] \\
\hline
\end{tabular}
\end{table}

\section{Standards et Références}
\label{app:standards}

\subsection{Standards Techniques Applicables}

\begin{itemize}
\item \textbf{[Standard 1]} - [Description et applicabilité au projet]
\item \textbf{[Standard 2]} - [Description et applicabilité au projet]
\item \textbf{[Standard 3]} - [Description et applicabilité au projet]
\item \textbf{[Standard 4]} - [Description et applicabilité au projet]
\item \textbf{[Standard 5]} - [Description et applicabilité au projet]
\end{itemize}

\subsection{Références Réglementaires}

\begin{itemize}
\item \textbf{[Réglementation 1]} - [Impact sur le projet]
\item \textbf{[Réglementation 2]} - [Impact sur le projet]
\item \textbf{[Réglementation 3]} - [Impact sur le projet]
\end{itemize}

\subsection{Documentation de Référence}

\begin{enumerate}
\item [Organisation] ([Année]). \textit{[Titre du document]}
\item [Organisation] ([Année]). \textit{[Titre du document]}
\item [Organisation] ([Année]). \textit{[Titre du document]}
\item [Organisation] ([Année]). \textit{[Titre du document]}
\item [Organisation] ([Année]). \textit{[Titre du document]}
\end{enumerate}

\section{Glossaire}
\label{app:glossary}

\begin{longtable}{|p{0.25\textwidth}|p{0.7\textwidth}|}
\caption{Glossaire des Termes Techniques} \label{tab:glossary} \\
\hline
\rowcolor{DollaramaGreen!30}
\textbf{\color{white}Terme} & \textbf{\color{white}Définition} \\
\hline
\endfirsthead

\multicolumn{2}{c}%
{{\tablename\ \thetable{} -- Suite de la page précédente}} \\
\hline
\rowcolor{DollaramaGreen!30}
\textbf{\color{white}Terme} & \textbf{\color{white}Définition} \\
\hline
\endhead

\hline \multicolumn{2}{|r|}{{Suite sur la page suivante}} \\ \hline
\endfoot

\hline
\endlastfoot

[Acronyme/Terme 1] & [Définition complète et contexte d'utilisation] \\
\hline
\rowcolor{gray!10}
[Acronyme/Terme 2] & [Définition complète et contexte d'utilisation] \\
\hline
[Acronyme/Terme 3] & [Définition complète et contexte d'utilisation] \\
\hline
\rowcolor{gray!10}
[Acronyme/Terme 4] & [Définition complète et contexte d'utilisation] \\
\hline
[Acronyme/Terme 5] & [Définition complète et contexte d'utilisation] \\
\hline
\end{longtable} in your main document

\clearpage
\appendix

% Reset page numbering for appendices
\pagenumbering{Roman}
\setcounter{page}{1}

\section{Spécifications Techniques Détaillées}
\label{app:technical-specs}

\subsection{[Composant Principal] - Spécifications}

\begin{table}[H]
\centering
\caption{Spécifications Techniques Complètes}
\label{tab:technical-specs}
\begin{tabular}{|p{0.3\textwidth}|p{0.6\textwidth}|}
\hline
\rowcolor{DollaramaGreen!30}
\textbf{\color{white}Composant} & \textbf{\color{white}Spécification Détaillée} \\
\hline
[Élément technique 1] & [Description détaillée, version, capacité] \\
\hline
\rowcolor{gray!10}
[Élément technique 2] & [Description détaillée, version, capacité] \\
\hline
[Élément technique 3] & [Description détaillée, version, capacité] \\
\hline
\rowcolor{gray!10}
[Élément technique 4] & [Description détaillée, version, capacité] \\
\hline
[Élément technique 5] & [Description détaillée, version, capacité] \\
\hline
\rowcolor{gray!10}
[Élément technique 6] & [Description détaillée, version, capacité] \\
\hline
[Certification/Conformité] & [Standards et certifications requis] \\
\hline
\end{tabular}
\end{table}

\subsection{Configuration Logicielle}

\begin{table}[H]
\centering
\caption{Stack Logiciel et Versions}
\label{tab:software-stack}
\begin{tabular}{|p{0.25\textwidth}|p{0.25\textwidth}|p{0.4\textwidth}|}
\hline
\rowcolor{DollaramaGreen!30}
\textbf{\color{white}Couche} & \textbf{\color{white}Composant} & \textbf{\color{white}Version/Configuration} \\
\hline
[Système d'exploitation] & [Nom] & [Version et configuration] \\
\hline
\rowcolor{gray!10}
[Base de données] & [Nom] & [Version et configuration] \\
\hline
[Serveur application] & [Nom] & [Version et configuration] \\
\hline
\rowcolor{gray!10}
[Framework] & [Nom] & [Version et configuration] \\
\hline
[Sécurité] & [Nom] & [Version et configuration] \\
\hline
\rowcolor{gray!10}
[Monitoring] & [Nom] & [Version et configuration] \\
\hline
[Backup] & [Nom] & [Version et configuration] \\
\hline
\end{tabular}
\end{table}

\section{Architecture Détaillée}
\label{app:architecture}

\subsection{Diagramme d'Architecture}

\begin{figure}[H]
\centering
\begin{tikzpicture}[node distance=3cm, auto, scale=0.8, transform shape]
    % Exemple de diagramme d'architecture
    % Remplacez par votre propre diagramme
    
    % Composants réseau
    \node [draw, rectangle, fill=DollaramaGreen!20, text width=2cm, text centered] (component1) {Composant 1};
    \node [draw, rectangle, below of=component1, fill=blue!20, text width=2cm, text centered] (component2) {Composant 2};
    \node [draw, rectangle, right of=component1, fill=DollaramaYellow!20, text width=2cm, text centered] (component3) {Composant 3};
    \node [draw, rectangle, below of=component3, fill=gray!20, text width=2cm, text centered] (component4) {Composant 4};
    
    % Connexions
    \draw [->] (component1) -- (component2);
    \draw [->] (component1) -- (component3);
    \draw [->] (component3) -- (component4);
    \draw [<->] (component2) -- (component4);
    
    % Annotations
    \node [text width=3cm] at ([xshift=4cm]component3.east) {Zone sécurisée};
\end{tikzpicture}
\caption{Architecture Système Proposée}
\label{fig:architecture}
\end{figure}

\textbf{Description:} [Expliquez les composants clés et les flux de données]

\section{Analyse Détaillée des Coûts}
\label{app:detailed-costs}

\subsection{Modèle de Coûts Complet}

\begin{longtable}{|p{0.3\textwidth}|p{0.15\textwidth}|p{0.15\textwidth}|p{0.15\textwidth}|p{0.2\textwidth}|}
\caption{Ventilation Détaillée des Coûts} \label{tab:detailed-costs} \\
\hline
\rowcolor{DollaramaGreen!30}
\textbf{\color{white}Catégorie} & 
\textbf{\color{white}Année 1} & 
\textbf{\color{white}Année 2} & 
\textbf{\color{white}Année 3} & 
\textbf{\color{white}Total 5 ans} \\
\hline
\endfirsthead

\multicolumn{5}{c}%
{{\tablename\ \thetable{} -- Suite de la page précédente}} \\
\hline
\rowcolor{DollaramaGreen!30}
\textbf{\color{white}Catégorie} & 
\textbf{\color{white}Année 1} & 
\textbf{\color{white}Année 2} & 
\textbf{\color{white}Année 3} & 
\textbf{\color{white}Total 5 ans} \\
\hline
\endhead

\hline \multicolumn{5}{|r|}{{Suite sur la page suivante}} \\ \hline
\endfoot

\hline
\endlastfoot

\textbf{COÛTS CAPEX} & & & & \\
\hline
[Matériel] & \$[Montant] & \$[Montant] & \$[Montant] & \$[Total] \\
\hline
\rowcolor{gray!10}
[Logiciels/Licences] & \$[Montant] & \$[Montant] & \$[Montant] & \$[Total] \\
\hline
[Infrastructure] & \$[Montant] & \$[Montant] & \$[Montant] & \$[Total] \\
\hline
\textbf{COÛTS OPEX} & & & & \\
\hline
[Support/Maintenance] & \$[Montant] & \$[Montant] & \$[Montant] & \$[Total] \\
\hline
\rowcolor{gray!10}
[Personnel] & \$[Montant] & \$[Montant] & \$[Montant] & \$[Total] \\
\hline
[Formation continue] & \$[Montant] & \$[Montant] & \$[Montant] & \$[Total] \\
\hline
\rowcolor{DollaramaGreen!20}
\textbf{TOTAL} & \textbf{\$[Total]} & \textbf{\$[Total]} & \textbf{\$[Total]} & \textbf{\$[TOTAL]} \\
\hline
\end{longtable}

\section{Documentation des Procédures}
\label{app:procedures}

\subsection{Procédure de [Nom du Processus]}

\begin{enumerate}
\item \textbf{[Étape 1 - Titre]}
    \begin{itemize}
    \item [Action spécifique à effectuer]
    \item [Vérification ou contrôle requis]
    \item [Personne responsable]
    \item [Temps estimé]
    \end{itemize}

\item \textbf{[Étape 2 - Titre]}
    \begin{itemize}
    \item [Action spécifique à effectuer]
    \item [Vérification ou contrôle requis]
    \item [Personne responsable]
    \item [Temps estimé]
    \end{itemize}

\item \textbf{[Étape 3 - Titre]}
    \begin{itemize}
    \item [Action spécifique à effectuer]
    \item [Vérification ou contrôle requis]
    \item [Personne responsable]
    \item [Temps estimé]
    \end{itemize}

\item \textbf{[Étape finale - Validation]}
    \begin{itemize}
    \item [Critères de validation]
    \item [Documentation à produire]
    \item [Approbations requises]
    \item [Archivage et communication]
    \end{itemize}
\end{enumerate}

\subsection{Guide d'Utilisation Rapide}

\begin{table}[H]
\centering
\caption{Procédure Simplifiée - Aide-Mémoire}
\label{tab:quick-guide}
\begin{tabular}{|p{0.3\textwidth}|p{0.65\textwidth}|}
\hline
\rowcolor{DollaramaGreen!30}
\textbf{\color{white}Étape} & \textbf{\color{white}Action} \\
\hline
1. [Titre court] & [Description de l'action en une phrase] \\
\hline
\rowcolor{gray!10}
2. [Titre court] & [Description de l'action en une phrase] \\
\hline
3. [Titre court] & [Description de l'action en une phrase] \\
\hline
\rowcolor{gray!10}
4. [Titre court] & [Description de l'action en une phrase] \\
\hline
5. [Titre court] & [Description de l'action en une phrase] \\
\hline
\rowcolor{gray!10}
6. [Titre court] & [Description de l'action en une phrase] \\
\hline
7. [Titre court] & [Description de l'action en une phrase] \\
\hline
\end{tabular}
\end{table}

\section{Standards et Références}
\label{app:standards}

\subsection{Standards Techniques Applicables}

\begin{itemize}
\item \textbf{[Standard 1]} - [Description et applicabilité au projet]
\item \textbf{[Standard 2]} - [Description et applicabilité au projet]
\item \textbf{[Standard 3]} - [Description et applicabilité au projet]
\item \textbf{[Standard 4]} - [Description et applicabilité au projet]
\item \textbf{[Standard 5]} - [Description et applicabilité au projet]
\end{itemize}

\subsection{Références Réglementaires}

\begin{itemize}
\item \textbf{[Réglementation 1]} - [Impact sur le projet]
\item \textbf{[Réglementation 2]} - [Impact sur le projet]
\item \textbf{[Réglementation 3]} - [Impact sur le projet]
\end{itemize}

\subsection{Documentation de Référence}

\begin{enumerate}
\item [Organisation] ([Année]). \textit{[Titre du document]}
\item [Organisation] ([Année]). \textit{[Titre du document]}
\item [Organisation] ([Année]). \textit{[Titre du document]}
\item [Organisation] ([Année]). \textit{[Titre du document]}
\item [Organisation] ([Année]). \textit{[Titre du document]}
\end{enumerate}

\section{Glossaire}
\label{app:glossary}

\begin{longtable}{|p{0.25\textwidth}|p{0.7\textwidth}|}
\caption{Glossaire des Termes Techniques} \label{tab:glossary} \\
\hline
\rowcolor{DollaramaGreen!30}
\textbf{\color{white}Terme} & \textbf{\color{white}Définition} \\
\hline
\endfirsthead

\multicolumn{2}{c}%
{{\tablename\ \thetable{} -- Suite de la page précédente}} \\
\hline
\rowcolor{DollaramaGreen!30}
\textbf{\color{white}Terme} & \textbf{\color{white}Définition} \\
\hline
\endhead

\hline \multicolumn{2}{|r|}{{Suite sur la page suivante}} \\ \hline
\endfoot

\hline
\endlastfoot

[Acronyme/Terme 1] & [Définition complète et contexte d'utilisation] \\
\hline
\rowcolor{gray!10}
[Acronyme/Terme 2] & [Définition complète et contexte d'utilisation] \\
\hline
[Acronyme/Terme 3] & [Définition complète et contexte d'utilisation] \\
\hline
\rowcolor{gray!10}
[Acronyme/Terme 4] & [Définition complète et contexte d'utilisation] \\
\hline
[Acronyme/Terme 5] & [Définition complète et contexte d'utilisation] \\
\hline
\end{longtable}