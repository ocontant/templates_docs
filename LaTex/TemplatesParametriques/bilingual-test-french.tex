\documentclass[french]{dollarama}
\usepackage{dollarama-business}
\usepackage{dollarama-layouts}

\title{Test de Modèle Bilingue}
\subtitle{Version Française}
\author{Développeur de Modèles}
\department{Architecture TI}
\status{Démo}
\version{1.0}
\classification{Interne}

\begin{document}
\maketitle

\executivesummary{Ceci est un énoncé de problème de test en français}{Ceci est la solution proposée en français}{Les avantages clés incluent le support multilingue, le formatage professionnel et l'image de marque corporative}{L'impact financier montre un système de modèles rentable}{Recommander l'implémentation du support bilingue pour tous les documents}

\section*{Section de Test}

Ce document démontre les capacités bilingues du système de modèles Dollarama.

\subheading{Caractéristiques Clés:}
\begin{itemize}
    \item Le résumé exécutif apparaît comme « Résumé Exécutif »
    \item Le pied de page montre « Confidentiel - Dollarama Inc. »
    \item Les étiquettes de la page titre sont en français
    \item Les jalons de chronologie utilisent le texte « Jalon »
\end{itemize}

\begin{dollaramatimeline}{Exemple de Chronologie}
    \milestone{Semaine 1}{Phase 1}{L'implémentation commence}
    \milestone{Semaine 2}{Phase 2}{Tests et validation}
\end{dollaramatimeline}

\end{document}