% EXEMPLE: Système Paramétrique Dollarama
% Démonstration du nouveau système de classes et packages
% Aucune modification des templates - tout est paramétrable!

\documentclass{TemplatesParametriques/dollarama}
\usepackage{TemplatesParametriques/dollarama-business}
\usepackage{TemplatesParametriques/dollarama-layouts}

% Métadonnées du document (paramètres de la classe)
\title{Modernisation du Système POS}
\subtitle{Initiative de Transformation Digitale}
\author{Marie Dupont}
\department{Technologies de l'Information}
\status{Final}
\version{2.0}
\classification{Confidentiel - Usage interne seulement}

\begin{document}

% Page de titre générée automatiquement avec les paramètres
\maketitle

% Table des matières avec style Dollarama
\tableofcontents
\clearpage

% ===============================================
% RÉSUMÉ EXÉCUTIF PARAMÉTRABLE
% ===============================================

\executivesummary{
    % Problématique
    Nos systèmes POS actuels datent de 2015 et causent des interruptions de service quotidiennes dans 45\% de nos magasins, générant des pertes de revenus estimées à 3.2M\$ annuellement.
}{
    % Solution proposée
    Déploiement d'une nouvelle infrastructure POS cloud-native avec capacités de paiement mobile et synchronisation temps réel, couvrant nos 1,600 magasins sur 18 mois.
}{
    % Bénéfices clés
    \begin{itemize}
        \item Réduction des temps d'arrêt de 89\% (de 4h/jour à 0.5h/jour)
        \item Amélioration de la vitesse de transaction de 60\%
        \item Support pour paiements mobiles et sans contact
        \item Intégration complète avec l'inventaire en temps réel
        \item Réduction des coûts de maintenance de 40\%
    \end{itemize}
}{
    % Impact financier
    \begin{itemize}
        \item \textbf{Investissement requis :} \$12.5M sur 18 mois
        \item \textbf{ROI estimé :} 156\% sur 5 ans
        \item \textbf{Économies annuelles :} \$4.8M dès la 2ème année
        \item \textbf{Période de récupération :} 31 mois
    \end{itemize}
}{
    % Recommandation
    Approuver immédiatement ce projet critique pour maintenir notre compétitivité et éviter les risques croissants de pannes système. Le démarrage en Q2 2025 permettrait une completion avant la saison des fêtes 2026.
}

% ===============================================
% ANALYSE AVEC LAYOUTS MODULAIRES
% ===============================================

\begin{dollaramaintroduction}{
    Dollarama opère le plus grand réseau de magasins à prix unique au Canada avec 1,600 points de vente. Notre infrastructure POS actuelle devient un facteur limitant majeur pour notre croissance et notre expérience client.
}
    Cette section présente l'analyse complète de notre situation actuelle et les recommandations pour moderniser notre infrastructure de point de vente.
    
    \highlightbox{
        \textbf{Urgence:} Le taux de panne de nos systèmes POS a augmenté de 340\% au cours des 12 derniers mois, nécessitant une action immédiate.
    }
\end{dollaramaintroduction}

% ===============================================
% COMPARAISON D'OPTIONS AVEC LAYOUT SPÉCIALISÉ
% ===============================================

\begin{dollaramacomparison}{Comparaison des Solutions POS}{Solution Évolutive}{Solution Complète}
    \textbf{Avantages:}
    \begin{itemize}
        \item Investissement initial réduit (\$4.2M)
        \item Déploiement par phases sur 3 ans
        \item Risque technique limité
        \item Compatibilité avec systèmes existants
    \end{itemize}
    
    \textbf{Inconvénients:}
    \begin{itemize}
        \item Bénéfices retardés
        \item Coûts de transition prolongés
        \item Maintien de la dette technique
    \end{itemize}
    
    \textbf{Coût total:} \$8.9M sur 5 ans
\vscompare
    \textbf{Avantages:}
    \begin{itemize}
        \item Transformation complète en 18 mois
        \item Bénéfices maximums dès la 2ème année
        \item Infrastructure moderne et évolutive
        \item Élimination de la dette technique
    \end{itemize}
    
    \textbf{Inconvénients:}
    \begin{itemize}
        \item Investissement initial important (\$12.5M)
        \item Risque de déploiement plus élevé
        \item Formation extensive requise
    \end{itemize}
    
    \textbf{Coût total:} \$12.5M sur 18 mois
\end{dollaramacomparison}

% ===============================================
% MÉTRIQUES FINANCIÈRES AVEC COMMANDES PARAMÉTRIQUES
% ===============================================

\begin{dollaramafinancials}{
    L'analyse financière démontre que malgré l'investissement initial important, la solution complète génère un retour sur investissement supérieur et des économies substantielles à long terme.
}
    \keyfinancialmetrics{\$31.2M}{\$156\%}{31 mois}{\$4.8M/an}
    
    \begin{dollaramacosttable}{Répartition des Coûts d'Investissement}{tab:pos-costs}
        \costrow{Matériel POS (terminaux et périphériques)}{1,600}{\$4,200}{\$6,720,000}
        \altcostrow{Logiciel et licences cloud}{1,600}{\$1,800}{\$2,880,000}
        \costrow{Installation et configuration}{1,600}{\$800}{\$1,280,000}
        \altcostrow{Formation du personnel}{3,200}{\$300}{\$960,000}
        \costrow{Gestion de projet et consultants}{1}{\$660,000}{\$660,000}
        \hline
        \totalrow{TOTAL INVESTISSEMENT}{12,500,000}
    \end{dollaramacosttable}
\end{dollaramafinancials}

% ===============================================
% PROCESSUS D'IMPLÉMENTATION AVEC LAYOUT STRUCTURÉ
% ===============================================

\begin{dollaramaprocess}{Processus d'Implémentation en 6 Étapes}
    \processstep{Planification et Acquisition}{
        \begin{itemize}
            \item Finalisation des spécifications techniques
            \item Processus d'appel d'offres et sélection fournisseur
            \item Commande des équipements pour phase pilote
            \item Préparation de l'infrastructure réseau
        \end{itemize}
        \textbf{Durée:} 3 mois | \textbf{Responsable:} Équipe Procurement
    }
    
    \processstep{Phase Pilote}{
        \begin{itemize}
            \item Déploiement dans 25 magasins représentatifs
            \item Tests d'intégration avec systèmes existants
            \item Formation des équipes pilotes
            \item Validation des processus opérationnels
        \end{itemize}
        \textbf{Durée:} 2 mois | \textbf{Responsable:} Équipe Technique
    }
    
    \processstep{Déploiement Régional}{
        \begin{itemize}
            \item Extension à 200 magasins par région
            \item Optimisation basée sur les apprentissages pilotes
            \item Formation des équipes régionales
            \item Support technique renforcé
        \end{itemize}
        \textbf{Durée:} 6 mois | \textbf{Responsable:} Équipes Régionales
    }
    
    \processstep{Déploiement National}{
        \begin{itemize}
            \item Déploiement des 1,375 magasins restants
            \item Processus industrialisé et optimisé
            \item Support 24/7 pendant la transition
            \item Monitoring intensif de la performance
        \end{itemize}
        \textbf{Durée:} 7 mois | \textbf{Responsable:} Équipe Nationale
    }
\end{dollaramaprocess}

% ===============================================
% GESTION DES RISQUES AVEC ENVIRONNEMENTS COLORÉS
% ===============================================

\begin{dollaramarisks}{
    Une approche proactive de gestion des risques est essentielle pour assurer le succès de cette transformation majeure. Nous avons identifié 12 risques principaux répartis en 4 catégories.
}
    \begin{dollaramariskcategory}{red}{Risques Techniques Critiques}
        \begin{itemize}
            \item \textbf{Incompatibilité systèmes:} Tests d'intégration approfondis prévus
            \item \textbf{Performance réseau:} Upgrade infrastructure réseau en parallèle
            \item \textbf{Sécurité données:} Audit de sécurité par tiers indépendant
        \end{itemize}
    \end{dollaramariskcategory}
    
    \begin{dollaramariskcategory}{orange}{Risques Opérationnels}
        \begin{itemize}
            \item \textbf{Résistance au changement:} Programme de change management
            \item \textbf{Formation insuffisante:} 40h de formation par employé
            \item \textbf{Interruption service:} Déploiement hors heures d'ouverture
        \end{itemize}
    \end{dollaramariskcategory}
    
    \begin{dollaramariskcategory}{DollaramaYellow}{Risques Financiers}
        \begin{itemize}
            \item \textbf{Dépassement budget:} Réserve de contingence 15\%
            \item \textbf{Délais de livraison:} Contrats avec pénalités
        \end{itemize}
    \end{dollaramariskcategory}
\end{dollaramarisks}

% ===============================================
% TIMELINE AVEC JALONS VISUELS
% ===============================================

\begin{dollaramatimeline}{
    Le projet s'étend sur 18 mois avec 6 jalons majeurs permettant une validation continue du progrès et des ajustements en cours de route.
}
    \milestone{Mars 2025}{Approbation et Financement}{Budget approuvé et équipe projet constituée}
    \milestone{Juin 2025}{Fin Phase Pilote}{25 magasins opérationnels et validés}
    \milestone{Septembre 2025}{Déploiement Régional 50\%}{200 magasins par région complétés}
    \milestone{Décembre 2025}{Déploiement National 25\%}{400 magasins nationaux complétés}
    \milestone{Juin 2026}{Déploiement Complet}{1,600 magasins opérationnels}
    \milestone{Septembre 2026}{Clôture Projet}{Transition vers opérations normales}
\end{dollaramatimeline}

% ===============================================
% RECOMMANDATIONS FINALES
% ===============================================

\begin{dollaramarecommendations}{
    Nous recommandons fortement l'approbation immédiate de la Solution Complète pour les raisons suivantes: ROI supérieur, élimination de la dette technique, et positionnement concurrentiel renforcé.
}
    \warningbox{
        \textbf{FACTEUR TEMPS CRITIQUE:} Chaque mois de retard coûte \$267,000 en pertes opérationnelles et retarde la récupération de l'investissement.
    }
    
    \textbf{Actions Immédiates Requises:}
    \begin{enumerate}
        \item Approbation du budget de \$12.5M par le conseil d'administration
        \item Nomination du chef de projet et constitution de l'équipe
        \item Lancement du processus d'appel d'offres
        \item Communication du projet aux parties prenantes
    \end{enumerate}
    
    \projectdashboard{18 mois}{\$12.5M}{6 jalons}{12 équipes}
\end{dollaramarecommendations}

\end{document}