% Test des en-têtes de tableaux avec contraste amélioré
\documentclass{article}
\usepackage[utf8]{inputenc}
\usepackage[T1]{fontenc}
\usepackage{xcolor}
\usepackage{colortbl}
\usepackage{TemplatesParametriques/dollarama-business}

% Définition des couleurs Dollarama
\definecolor{DollaramaGreen}{HTML}{00A651}
\definecolor{DollaramaYellow}{HTML}{FFF200}
\definecolor{DollaramaDarkGreen}{HTML}{008A42}

\begin{document}

\title{Test des En-têtes de Tableaux}
\author{Test}
\date{\today}
\maketitle

\section{Tableau avec En-têtes Standards (Contraste Amélioré)}

\begin{table}[h]
\centering
\caption{Exemple avec en-têtes standards - DollaramaGreen 80\% (amélioration du contraste)}
\begin{tabular}{|l|r|r|r|}
\hline
\dollaramatabheader
\dollaramatabheadertext{Produit} & 
\dollaramatabheadertext{Quantité} & 
\dollaramatabheadertext{Prix Unit.} & 
\dollaramatabheadertext{Total} \\
\hline
Produit A & 100 & \$5.00 & \$500.00 \\
\hline
Produit B & 150 & \$3.50 & \$525.00 \\
\hline
\end{tabular}
\end{table}

\section{Tableau avec En-têtes Alternatifs (Jaune)}

\begin{table}[h]
\centering
\caption{Exemple avec en-têtes alternatifs - DollaramaYellow 90\%}
\begin{tabular}{|l|r|r|r|}
\hline
\dollaramatabheaderalt
\dollaramatabheaderaltext{Critère} & 
\dollaramatabheaderaltext{Option A} & 
\dollaramatabheaderaltext{Option B} & 
\dollaramatabheaderaltext{Différence} \\
\hline
Coût & \$1000 & \$1200 & \$200 \\
\hline
Performance & 95\% & 90\% & -5\% \\
\hline
\end{tabular}
\end{table}

\section{Tableau avec En-têtes Critiques (Rouge)}

\begin{table}[h]
\centering
\caption{Exemple avec en-têtes critiques - Rouge 80\%}
\begin{tabular}{|l|c|c|c|}
\hline
\dollaramatabheadercritical
\dollaramatabheadertext{Risque} & 
\dollaramatabheadertext{Impact} & 
\dollaramatabheadertext{Probabilité} & 
\dollaramatabheadertext{Score} \\
\hline
Risque A & Élevé & Moyen & 0.6 \\
\hline
Risque B & Faible & Élevé & 0.3 \\
\hline
\end{tabular}
\end{table}

\section{Comparaison Avant/Après}

\textbf{Amélioration du contraste:}
\begin{itemize}
\item \textbf{Avant:} DollaramaGreen!30 (30\% d'opacité) - Contraste insuffisant
\item \textbf{Après:} DollaramaGreen!80 (80\% d'opacité) - Contraste conforme aux standards UX/UI
\end{itemize}

\textbf{Avantages du nouveau système:}
\begin{itemize}
\item Meilleure lisibilité et accessibilité
\item Respect des standards de contraste WCAG
\item Commandes réutilisables standardisées
\item Cohérence visuelle garantie
\end{itemize}

\end{document}