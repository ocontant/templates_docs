% EXEMPLE: Analyse de Risques PMP - Version Simple
% Démonstration du module de gestion des risques PMP

\documentclass{TemplatesParametriques/dollarama}
\usepackage{TemplatesParametriques/dollarama-business}
\usepackage{TemplatesParametriques/dollarama-pmp-risks}

% Configuration du projet
\title{Analyse de Risques PMP}
\subtitle{Projet de Modernisation Infrastructure Cloud}
\author{Sophie Lavoie, PMP}
\department{PMO - Bureau de Projets}
\status{Version Finale}
\version{3.0}
\classification{Confidentiel - Comité Exécutif}

\begin{document}

% Page de titre automatique
\maketitle

% Table des matières
\tableofcontents
\clearpage

% ===============================================
% INTRODUCTION
% ===============================================

\section{Introduction}

Ce projet vise à migrer 85\% de notre infrastructure IT vers le cloud Azure sur 24 mois, avec un budget de 15.2M\$ et l'implication de 12 équipes techniques. Cette analyse suit les standards PMP du PMI.

\infobox{
    \textbf{Méthodologie PMP:} Cette analyse utilise les processus 11.1 à 11.7 du PMBOK Guide et respecte les standards ISO 31000 pour la gestion des risques.
}

% ===============================================
% MATRICE DE RISQUE PMP
% ===============================================

\section{Matrice de Risque PMP}

\pmpriskmatrix

\highlightbox{
    \textbf{Seuils de Tolérance Dollarama:}
    \begin{itemize}
        \item \textbf{Risques Critiques (>0.5):} Escalade immédiate au sponsor
        \item \textbf{Risques Élevés (0.3-0.5):} Réunion hebdomadaire de mitigation
        \item \textbf{Risques Moyens (0.1-0.3):} Surveillance bi-mensuelle
        \item \textbf{Risques Faibles (<0.1):} Revue trimestrielle
    \end{itemize}
}

% ===============================================
% REGISTRE DES RISQUES
% ===============================================

\begin{pmpriskreregister}{
    Projet de migration cloud critique pour la transformation digitale de Dollarama. Budget: 15.2M\$, durée: 24 mois, 12 équipes impliquées, 1,600 points de vente affectés.
}

\riskentry{R001}{Indisponibilité prolongée pendant migration}{\probmedium}{\impactveryhigh}{\pmpriskhigh{0.24}}{\mitigate{Migration par lots avec rollback automatique}}
\rowcolor{gray!10}
\riskentry{R002}{Dépassement budgétaire complexité sous-estimée}{\probhigh}{\impacthigh}{\pmpriskmedium{0.20}}{\mitigate{Réserve 15\% + suivi hebdomadaire}}
\riskentry{R003}{Résistance changement équipes techniques}{\probveryhigh}{\impactmedium}{\pmpriskmedium{0.14}}{\mitigate{Formation intensive + champions}}
\rowcolor{gray!10}
\riskentry{R004}{Problèmes performance applications cloud}{\probhigh}{\impacthigh}{\pmpriskmedium{0.20}}{\mitigate{Tests charge + optimisation}}
\riskentry{R005}{Faille sécurité migration données}{\probverylow}{\impactveryhigh}{\pmprisklow{0.08}}{\transfer{Assurance cyber + audit tiers}}
\rowcolor{gray!10}
\riskentry{R006}{Retard fournisseur cloud}{\probmedium}{\impacthigh}{\pmpriskmedium{0.12}}{\mitigate{Clauses pénalités + plan B}}
\riskentry{R007}{Perte données critiques migration}{\probverylow}{\impactveryhigh}{\pmprisklow{0.08}}{\avoid{Triple backup + tests restauration}}
\rowcolor{gray!10}
\riskentry{R008}{Manque compétences cloud équipe IT}{\probveryhigh}{\impactmedium}{\pmpriskmedium{0.14}}{\mitigate{Certification Azure + consultants}}
\riskentry{R009}{Opportunité économies optimisation}{\probhigh}{\impacthigh}{\pmpriskmedium{0.24}}{\exploit{Analyse proactive + renégociation}}
\rowcolor{gray!10}
\riskentry{R010}{Changement réglementaire stockage cloud}{\probmedium}{\impacthigh}{\pmprisklow{0.08}}{\accept{Surveillance légale + conformité}}

\end{pmpriskreregister}

% ===============================================
% TABLEAU DE BORD RISQUES
% ===============================================

\section{Tableau de Bord des Risques}

\riskdashboardpmp{0}{4}{4}{2}{1}

\warningbox{
    \textbf{ALERTES ACTUELLES:}
    \begin{itemize}
        \item 4 risques élevés nécessitent une attention soutenue
        \item Budget de contingence à 85\% d'utilisation
        \item 3 plans de mitigation en retard sur l'échéancier
        \item 1 opportunité non exploitée identifiée
    \end{itemize}
}

% ===============================================
% ANALYSE QUANTITATIVE
% ===============================================

\section{Analyse Quantitative}

\quantitativeriskanalysis{3,200,000}{2,500,000}{DÉPASSÉ}{3,800,000}

\begin{dollaramacashflowtable}{Impact Financier des Risques Principaux}{tab:risk-impact}
\textbf{Scénario de Base} & \$0 & \$15,200,000 & \$0 & \$0 & \$0 & \$0 \\
\hline
\rowcolor{gray!10}
R001 - Indisponibilité (24\%) & \$0 & \$0 & \$1,200,000 & \$0 & \$0 & \$0 \\
\hline
R002 - Dépassement budget (20\%) & \$0 & \$2,800,000 & \$0 & \$0 & \$0 & \$0 \\
\hline
\rowcolor{gray!10}
R004 - Performance (20\%) & \$0 & \$0 & \$800,000 & \$400,000 & \$0 & \$0 \\
\hline
R009 - Opportunité (24\%) & \$0 & -\$600,000 & -\$800,000 & -\$800,000 & -\$800,000 & -\$800,000 \\
\hline
\rowcolor{DollaramaGreen!20}
\textbf{Exposition Totale} & \$0 & \$17,400,000 & \$1,200,000 & -\$400,000 & -\$800,000 & -\$800,000 \\
\hline
\end{dollaramacashflowtable}

% ===============================================
% ANALYSE SWOT
% ===============================================

\section{Analyse SWOT}

\swotanalysis{
    \textbf{Forces Internes:}
    \begin{itemize}
        \item Équipe IT expérimentée (15 ans moyenne)
        \item Infrastructure réseau robuste
        \item Support exécutif fort et engagé
        \item Budget approuvé et disponible
    \end{itemize}
}{
    \textbf{Faiblesses Internes:}
    \begin{itemize}
        \item Manque d'expertise cloud native
        \item Systèmes legacy complexes et interdépendants
        \item Résistance culturelle au changement
        \item Documentation technique incomplète
    \end{itemize}
}{
    \textbf{Opportunités Externes:}
    \begin{itemize}
        \item Économies d'échelle cloud significatives
        \item Nouvelles fonctionnalités Azure disponibles
        \item Amélioration drastique de la sécurité
        \item Flexibilité et scalabilité illimitées
    \end{itemize}
}{
    \textbf{Menaces Externes:}
    \begin{itemize}
        \item Changements réglementaires imprévisibles
        \item Cyberattaques de plus en plus sophistiquées
        \item Dépendance critique à un fournisseur unique
        \item Volatilité et augmentation des prix cloud
    \end{itemize}
}

% ===============================================
% PLANS DE RÉPONSE
% ===============================================

\section{Plans de Réponse aux Risques}

\riskresponseplan{R001 - Indisponibilité Services}{
    \mitigate{Stratégie d'atténuation proactive avec rollback automatique}
}{
    \begin{itemize}
        \item Migration par lots de 50 magasins maximum
        \item Tests de basculement automatique avant chaque lot
        \item Équipe de rollback disponible 24/7 pendant migrations
        \item Monitoring en temps réel des performances critiques
        \item Communication proactive avec tous les magasins
    \end{itemize}
}{
    Marc Dubois, Architecte Cloud Senior
}{
    Plan actif jusqu'à completion migration (Décembre 2026)
}

\riskresponseplan{R002 - Dépassement Budgétaire}{
    \mitigate{Contrôle strict des coûts avec contingence robuste}
}{
    \begin{itemize}
        \item Réserve de contingence de 15\% (\$2.3M) préapprouvée
        \item Suivi budgétaire hebdomadaire avec reporting exécutif
        \item Approbation obligatoire pour tout dépassement >50K\$
        \item Renégociation proactive des contrats si nécessaire
        \item Priorisation dynamique des fonctionnalités critiques
    \end{itemize}
}{
    Louise Martin, Contrôleur Financier
}{
    Surveillance continue pendant toute la durée du projet
}

\riskresponseplan{R009 - Opportunité d'Économies Supplémentaires}{
    \exploit{Maximisation proactive des bénéfices inattendus}
}{
    \begin{itemize}
        \item Analyse mensuelle détaillée des coûts cloud
        \item Implémentation d'optimisation automatique des ressources
        \item Renégociation trimestrielle des tarifs avec Microsoft
        \item Déploiement stratégique de Reserved Instances
        \item Partage des économies réalisées avec les départements
    \end{itemize}
}{
    Sarah Thompson, Gestionnaire Cloud
}{
    Révision trimestrielle des opportunités avec comité directeur
}

% ===============================================
% PROCESSUS DE SURVEILLANCE
% ===============================================

\section{Processus de Surveillance}

\textbf{Fréquence de Surveillance:}
\begin{itemize}
    \item \textbf{Risques Critiques:} Surveillance quotidienne avec rapport
    \item \textbf{Risques Élevés:} Revue hebdomadaire avec actions
    \item \textbf{Risques Moyens:} Revue bi-mensuelle avec tendances
    \item \textbf{Risques Faibles:} Revue mensuelle avec statut global
\end{itemize}

\textbf{Mécanismes d'Escalade:}
\begin{enumerate}
    \item \textbf{Niveau 1:} Chef de projet (résolution < 24h)
    \item \textbf{Niveau 2:} Sponsor projet (résolution < 72h)
    \item \textbf{Niveau 3:} Comité directeur (résolution < 1 semaine)
    \item \textbf{Niveau 4:} Comité exécutif (décision stratégique)
\end{enumerate}

% ===============================================
% RECOMMANDATIONS
% ===============================================

\section{Recommandations}

Basé sur cette analyse PMP complète, le projet présente un profil de risque \textbf{MOYEN} avec des opportunités substantielles. Les 4 risques élevés identifiés ont des plans de mitigation solides et un financement de contingence approprié.

\textbf{Actions Prioritaires:}
\begin{enumerate}
    \item Activation immédiate des plans de mitigation pour R001, R002, R003 et R004
    \item Augmentation de la réserve de contingence à 18\% (\$2.7M)
    \item Mise en place d'un centre de commande 24/7 pendant la migration
    \item Formation accélérée des équipes sur Azure (certification obligatoire)
    \item Signature d'un contrat d'assurance cyber-risques spécialisé
\end{enumerate}

\warningbox{
    \textbf{DÉCISION REQUISE:} Approbation du budget de contingence supplémentaire (\$400K) avant le démarrage de la phase de migration critique en Mars 2025.
}

\textbf{Probabilité de Succès du Projet:} 87\% (dans les paramètres de coût et délai approuvés)

\end{document}