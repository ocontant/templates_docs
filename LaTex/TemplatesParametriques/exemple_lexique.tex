\documentclass{dollarama}

% Définition des termes techniques
\defterm{NVR}{Network Video Recorder}
\defterm{DVR}{Digital Video Recorder}
\defterm{API}{Application Programming Interface}
\defterm{IoT}{Internet of Things}
\defterm{VPN}{Virtual Private Network}

\title{Système de Sécurité Dollarama}
\subtitle{Documentation Technique avec Footnotes}
\author{Équipe Sécurité}
\date{\today}
\department{Technologies de l'Information}
\status{Confidentiel}
\version{1.0}

\begin{document}

\maketitle

\section{Introduction}

Dans le cadre de notre infrastructure de sécurité, nous utilisons des \term{NVR} et des \term{DVR} pour l'enregistrement vidéo. Ces systèmes sont connectés via notre \term{API} interne et intégrés avec nos dispositifs \term{IoT}.

L'accès à distance se fait via \term{VPN} sécurisé pour maintenir la confidentialité des données.

\newpage

\section{Configuration Réseau}

Les \term{NVR} sont configurés pour communiquer avec les caméras IP via des protocoles standardisés. Chaque \term{DVR} dispose d'une interface d'administration accessible via l'\term{API} REST.

Nous pouvons également utiliser des footnotes normales\footnote{Ceci est une footnote standard} en parallèle avec les termes techniques.

\newpage

\section{Maintenance}

Pour la maintenance des systèmes \term{IoT} et l'accès via \term{VPN}, suivre les procédures standard.

% Exemple avec légende personnalisée si besoin
\customlegend[Note Importante]{
Ce document utilise maintenant le système de footnotes avec numérotation superscript automatique pour les termes techniques.
}

\end{document}