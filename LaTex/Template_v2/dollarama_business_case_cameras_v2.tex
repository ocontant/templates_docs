% Dollarama Business Case - Infrastructure de Vidéosurveillance
% Recreated using the new parametric template system with annual report layout integration
% Original: Olivier Contant - Concepteur Infrastructure
% Updated: Template System v2.0

\documentclass{dollarama}

% Import parametric business modules
\usepackage{dollarama-business}
\usepackage{dollarama-layouts}
\usepackage{dollarama-pmp-risks}

% Import new annual report layout system
\usepackage{dollarama-annual-report-layout}
\usepackage{dollarama-sections-all}

% Typography optimization for French content
\hyphenpenalty=8000
\exhyphenpenalty=8000
\tolerance=2000
\emergencystretch=\maxdimen
\hbadness=10000

% Page setup for full-page title
\geometry{
    top=0cm,
    bottom=0cm,
    left=0cm,
    right=0cm,
    showframe=false
}

% Configuration des métadonnées
\title{Business Case}
\subtitle{Modernisation de l'Infrastructure de Vidéosurveillance}
\author{Olivier Contant - Concepteur Infrastructure}
\department{Projet Infrastructure}
\status{Draft}
\version{0.1}
\classification{Confidentiel - Usage interne}

% Set hero image for title page
\setheroimage{images/dollarama_alt2.png}

\begin{document}

% Page de titre automatique avec nouveau design
\maketitle

% Reset geometry for content pages
\newgeometry{
    top=5cm,
    bottom=2cm,
    left=1.5cm,
    right=2cm
}

% Table des matières
\tableofcontents
\clearpage

% ===============================================
% EXECUTIVE SUMMARY WITH NEW TEMPLATE DESIGN
% ===============================================

\dollaramasectionintroduction{Sommaire Exécutif}{
    Vue d'ensemble stratégique du projet de modernisation
}{
    Le projet de modernisation de l'infrastructure de vidéosurveillance de Dollarama représente un investissement stratégique critique pour améliorer la sécurité, réduire les pertes et optimiser les opérations dans l'ensemble du réseau de magasins.

    Cette analyse de rentabilisation démontre les avantages substantiels de la mise à niveau vers un système de surveillance moderne et intégré.
}{images/dollarama_hq_exterior.jpg}

\begin{executivesummaryenv}

\execsection{Problématique Actuelle}
Le système de vidéosurveillance existant présente des limitations majeures : équipements obsolètes, qualité d'image insuffisante, absence d'intégration centralisée et capacités d'analyse limitées. Ces déficiences exposent Dollarama à des risques accrus de pertes, de responsabilité légale et d'inefficacités opérationnelles.

\execsection{Solution Proposée}
Déploiement d'une infrastructure de vidéosurveillance de nouvelle génération comprenant :
\begin{itemize}
    \item Caméras haute définition IP avec analyse intelligente
    \item Système de gestion centralisé basé sur le cloud
    \item Solutions d'analyse comportementale et de détection automatique
    \item Intégration avec les systèmes POS et de sécurité existants
\end{itemize}

\execsection{Bénéfices Attendus}
\begin{itemize}
    \item \textbf{Réduction des pertes :} 15-25\% de diminution du vol à l'étalage
    \item \textbf{Efficacité opérationnelle :} Amélioration de 20\% des processus de sécurité
    \item \textbf{Conformité réglementaire :} Respect des standards de l'industrie
    \item \textbf{Retour sur investissement :} Payback de 18 mois
\end{itemize}

\execsection{Investissement Requis}
Investissement total estimé à \currency{4,500,000} réparti sur 24 mois, avec un retour sur investissement projeté de \percentage{28} la première année.

\end{executivesummaryenv}

\clearpage

% ===============================================
% BUSINESS CONTEXT WITH FINANCIAL TEMPLATE
% ===============================================

\dollaramasectionfinancial{Contexte d'Affaires}{
    Analyse de l'environnement et des enjeux stratégiques
}{
    Le secteur du commerce de détail fait face à des défis croissants en matière de sécurité et de gestion des pertes. Pour Dollarama, leader canadien du commerce à prix unique, la protection des actifs et l'optimisation opérationnelle sont essentielles au maintien de la rentabilité et de la croissance.
}{images/dollarama_hq_office.jpg}

\dollaramarevenueanalysis{Enjeux Stratégiques Actuels}{
    \begin{itemize}
        \item \textbf{Pertes Inventaire :} 2.1\% du chiffre d'affaires (moyenne industrie : 1.8\%)
        \item \textbf{Incidents Sécurité :} Augmentation de 12\% sur les 24 derniers mois
        \item \textbf{Coûts Opérationnels :} Surveillance manuelle représentant 15\% du budget sécurité
        \item \textbf{Conformité :} Nouvelles exigences réglementaires en matière de protection des données
    \end{itemize}
}

\highlightbox{
\textbf{Impact Business Critical :} L'infrastructure de surveillance actuelle limite la capacité de Dollarama à détecter et prévenir les pertes, affectant directement la marge bénéficiaire et la sécurité des employés et clients.
}

\keyfinancialmetrics{\$4.5M}{28\%}{18 mois}{1,650 magasins}

\clearpage

% ===============================================
% PROBLEM DEFINITION WITH STRATEGIC TEMPLATE
% ===============================================

\dollaramasectionstrategic{Définition du Problème}{
    Analyse des limitations et risques actuels
}{
    L'infrastructure de vidéosurveillance existante de Dollarama présente des déficiences majeures qui compromettent l'efficacité opérationnelle et exposent l'organisation à des risques significatifs.
}{images/header_retail_aisle.jpg}

\begin{dollaramarisks}{Analyse des Risques Actuels}

\begin{dollaramariskcategory}{red}{Risques Technologiques}
\begin{itemize}
    \item Équipements obsolètes (80\% des caméras ont plus de 7 ans)
    \item Qualité d'image insuffisante pour identification
    \item Absence de redondance système
    \item Capacités de stockage limitées (7 jours maximum)
\end{itemize}
\end{dollaramariskcategory}

\begin{dollaramariskcategory}{orange}{Risques Opérationnels}
\begin{itemize}
    \item Surveillance manuelle intensive requise
    \item Temps de réponse aux incidents élevé
    \item Formation extensive nécessaire pour personnel
    \item Maintenance corrective fréquente
\end{itemize}
\end{dollaramariskcategory}

\begin{dollaramariskcategory}{yellow}{Risques Financiers}
\begin{itemize}
    \item Pertes par vol non détecté
    \item Coûts de maintenance croissants
    \item Responsabilité légale en cas d'incident
    \item Inefficacités dans allocation des ressources sécurité
\end{itemize}
\end{dollaramariskcategory}

\end{dollaramarisks}

\infobox{
\textbf{Évaluation d'Impact :} Une analyse quantitative révèle que les limitations actuelles coûtent approximativement \currency{2,800,000} annuellement en pertes directes et indirectes, justifiant largement l'investissement dans la modernisation.
}

\clearpage

% ===============================================
% SOLUTIONS ANALYSIS WITH COMPARISON FRAMEWORK
% ===============================================

\dollaramasectionstrategic{Analyse des Solutions}{
    Évaluation comparative des options technologiques
}{
    Trois approches ont été évaluées pour la modernisation de l'infrastructure de vidéosurveillance, chacune présentant des avantages et considérations spécifiques.
}{images/header_retail_aisle.jpg}

\begin{dollaramacomparison}{Comparaison des Options Technologiques}

\textbf{Option 1 : Mise à Niveau Graduelle}
\begin{itemize}
    \item Remplacement par phases sur 36 mois
    \item Conservation partielle de l'infrastructure existante
    \item Investissement initial réduit : \currency{2,800,000}
    \item Risque de compatibilité et performance limitée
\end{itemize}

\vscompare

\textbf{Option 2 : Remplacement Complet (Recommandée)}
\begin{itemize}
    \item Déploiement intégral sur 24 mois
    \item Infrastructure entièrement nouvelle et harmonisée
    \item Investissement optimal : \currency{4,500,000}
    \item Performance maximale et évolutivité garantie
\end{itemize}

\end{dollaramacomparison}

\vspace{1cm}

\begin{dollaramacomparison}{Analyse Technologique Détaillée}

\textbf{Technologies Actuelles}
\begin{itemize}
    \item Caméras analogiques 720p
    \item Enregistreurs DVR locaux
    \item Surveillance manuelle
    \item Stockage local limité
    \item Pas d'analyse automatique
\end{itemize}

\vscompare

\textbf{Technologies Proposées}
\begin{itemize}
    \item Caméras IP 4K avec IA intégrée
    \item Plateforme cloud centralisée
    \item Analyse comportementale automatique
    \item Stockage cloud extensible
    \item Alertes en temps réel
\end{itemize}

\end{dollaramacomparison}

\warningbox{
\textbf{Recommandation Stratégique :} L'Option 2 (Remplacement Complet) est fortement recommandée pour maximiser le retour sur investissement et assurer la pérennité technologique sur 7-10 ans.
}

\clearpage

% ===============================================
% IMPACT ANALYSIS WITH FINANCIAL DASHBOARD
% ===============================================

\dollaramasectionfinancial{Analyse d'Impact}{
    Évaluation quantitative des bénéfices
}{
    L'implémentation de la solution recommandée génèrera des bénéfices mesurables dans multiple domaines opérationnels et financiers.
}{images/dollarama_hq_office.jpg}

\begin{dollaramafinancials}{
    Analyse détaillée de l'impact financier et opérationnel de la modernisation de l'infrastructure de vidéosurveillance.
}
    \keyfinancialmetrics{\$6.2M}{\$4.5M}{18 mois}{28\%}
\end{dollaramafinancials}

\begin{dollaramacosttable}{Analyse Coûts-Bénéfices}{tab:cost-benefit}
\dollaramatabheader
\dollaramatabheadertext{Catégorie} & \dollaramatabheadertext{Année 1} & \dollaramatabheadertext{Année 2} & \dollaramatabheadertext{Année 3} \\
\hline
\textbf{Réduction Pertes} & \currency{1,200,000} & \currency{1,350,000} & \currency{1,400,000} \\
\textbf{Efficacité Opérationnelle} & \currency{450,000} & \currency{520,000} & \currency{580,000} \\
\textbf{Réduction Maintenance} & \currency{180,000} & \currency{200,000} & \currency{220,000} \\
\textbf{Conformité/Assurance} & \currency{120,000} & \currency{140,000} & \currency{160,000} \\
\hline
\dollaramatabheaderalt
\dollaramatabheaderaltext{Total Bénéfices} & \dollaramatabheaderaltext{\currency{1,950,000}} & \dollaramatabheaderaltext{\currency{2,210,000}} & \dollaramatabheaderaltext{\currency{2,360,000}} \\
\end{dollaramacosttable}

\clearpage

% ===============================================
% RISK MANAGEMENT WITH PMP FRAMEWORK
% ===============================================

\dollaramasectionstrategic{Matrice des Risques}{
    Analyse PMI et stratégies d'atténuation
}{
    Une analyse complète des risques projet selon les standards PMI identifie les menaces potentielles et les stratégies d'atténuation appropriées.
}{images/header_retail_aisle.jpg}

\begin{dollaramacosttable}{Registre des Risques Projet}{tab:risk-register}
    \costrow{R001: Dépassement budgétaire}{Faible}{Moyen}{Contingence 10\%, contrôle hebdomadaire}
    \altcostrow{R002: Retard livraison équipements}{Moyen}{Élevé}{Fournisseurs multiples, commandes anticipées}
    \costrow{R003: Résistance changement utilisateurs}{Élevé}{Faible}{Formation renforcée, communication continue}
    \altcostrow{R004: Problèmes intégration systèmes}{Moyen}{Moyen}{Tests préalables, expertise technique}
    \costrow{R005: Violation données sécurité}{Faible}{Très Élevé}{Chiffrement, contrôles accès stricts}
\end{dollaramacosttable}

\vspace{1cm}

\infobox{
\textbf{Stratégie Globale de Gestion des Risques :} Approche proactive avec monitoring continu, plans de contingence détaillés et escalation claire pour tous risques de niveau médium et élevé.
}

\clearpage

% ===============================================
% COST ESTIMATION WITH DETAILED BREAKDOWN
% ===============================================

\dollaramasectionfinancial{Estimation des Coûts}{
    Analyse financière détaillée
}{
    L'investissement total de \currency{4,500,000} se répartit stratégiquement sur les composantes technologiques, d'implémentation et de formation pour maximiser la valeur et minimiser les risques.
}{images/dollarama_hq_office.jpg}

\begin{dollaramacashflowtable}{Projection Flux de Trésorerie - 6 Ans}{tab:cashflow}
\dollaramatabheader
\dollaramatabheadertext{Composante} & \dollaramatabheadertext{An 0} & \dollaramatabheadertext{An 1} & \dollaramatabheadertext{An 2} & \dollaramatabheadertext{An 3} & \dollaramatabheadertext{An 4} & \dollaramatabheadertext{An 5} \\
\hline
\textbf{Investissement Initial} & \negative{\currency{2,700,000}} & \negative{\currency{1,800,000}} & \currency{0} & \currency{0} & \currency{0} & \currency{0} \\
\textbf{Bénéfices Opérationnels} & \currency{0} & \positive{\currency{1,950,000}} & \positive{\currency{2,210,000}} & \positive{\currency{2,360,000}} & \positive{\currency{2,480,000}} & \positive{\currency{2,590,000}} \\
\textbf{Coûts Maintenance} & \currency{0} & \negative{\currency{180,000}} & \negative{\currency{195,000}} & \negative{\currency{210,000}} & \negative{\currency{225,000}} & \negative{\currency{240,000}} \\
\hline
\dollaramatabheaderalt
\dollaramatabheaderaltext{Flux Net} & \dollaramatabheaderaltext{\negative{\currency{2,700,000}}} & \dollaramatabheaderaltext{\negative{\currency{30,000}}} & \dollaramatabheaderaltext{\positive{\currency{2,015,000}}} & \dollaramatabheaderaltext{\positive{\currency{2,150,000}}} & \dollaramatabheaderaltext{\positive{\currency{2,255,000}}} & \dollaramatabheaderaltext{\positive{\currency{2,350,000}}} \\
\dollaramatabheaderalt
\dollaramatabheaderaltext{Flux Cumulé} & \dollaramatabheaderaltext{\negative{\currency{2,700,000}}} & \dollaramatabheaderaltext{\negative{\currency{2,730,000}}} & \dollaramatabheaderaltext{\negative{\currency{715,000}}} & \dollaramatabheaderaltext{\positive{\currency{1,435,000}}} & \dollaramatabheaderaltext{\positive{\currency{3,690,000}}} & \dollaramatabheaderaltext{\positive{\currency{6,040,000}}} \\
\end{dollaramacashflowtable}

\projectdashboard{24 mois}{\currency{4.5M}}{12 étapes}{85 ressources}

\clearpage

% ===============================================
% RECOMMENDATIONS WITH IMPLEMENTATION FOCUS
% ===============================================

\dollaramasectionconclusion{Recommandations}{
    Plan d'action stratégique
}{
    Basé sur l'analyse complète des coûts, bénéfices et risques, nous recommandons l'approbation immédiate du projet de modernisation avec démarrage au Q1 2025.
}{images/dollarama_hq_atrium.jpg}

\begin{dollaramarecommendations}{
    Recommandation d'approbation avec déploiement selon la méthodologie PMI pour garantir la réussite du projet et la réalisation des bénéfices attendus.
}
    \projectdashboard{Q1 2025}{\currency{4.5M}}{Phase 1}{Approuvé}
\end{dollaramarecommendations}

\dollaramastrategicrecommendations{
\textbf{Actions Immédiates Recommandées :}
\begin{enumerate}
    \item \textbf{Approbation Budgétaire :} Autorisation de l'investissement de \currency{4,500,000}
    \item \textbf{Constitution Équipe :} Formation de l'équipe projet avec PMO dédié
    \item \textbf{Sélection Fournisseurs :} Lancement appel d'offres pour intégrateurs qualifiés
    \item \textbf{Planification Détaillée :} Élaboration du planning d'implémentation par magasin
\end{enumerate}

\textbf{Facteurs Critiques de Succès :}
\begin{itemize}
    \item Support exécutif visible et continu
    \item Communication transparente avec équipes opérationnelles
    \item Formation complète du personnel avant déploiement
    \item Tests pilotes dans magasins sélectionnés
    \item Monitoring performance post-implémentation
\end{itemize}
}

\clearpage

% ===============================================
% IMPLEMENTATION PLAN WITH TIMELINE
% ===============================================

\dollaramasectionstrategic{Plan d'Implémentation}{
    Roadmap détaillée de déploiement
}{
    Le déploiement s'effectuera selon une approche phasée sur 24 mois, permettant d'optimiser les ressources et minimiser les disruptions opérationnelles.
}{images/header_retail_aisle.jpg}

\begin{dollaramatimeline}{Jalons Critiques du Projet}

\milestone{Q1 2025}{Phase 1 - Conception et Préparation}{
    \begin{itemize}
        \item Finalisation architecture technique
        \item Sélection et contractualisation fournisseurs
        \item Formation équipe projet
        \item Préparation sites pilotes (5 magasins)
    \end{itemize}
}

\milestone{Q2 2025}{Phase 2 - Déploiement Pilote}{
    \begin{itemize}
        \item Installation complète sur sites pilotes
        \item Tests d'intégration et validation performances
        \item Formation utilisateurs pilotes
        \item Ajustements procédures basés sur retours
    \end{itemize}
}

\milestone{Q3-Q4 2025}{Phase 3 - Déploiement Régional}{
    \begin{itemize}
        \item Rollout sur 40\% du réseau (660 magasins)
        \item Support technique intensif
        \item Monitoring performances en continu
        \item Optimisations basées sur données opérationnelles
    \end{itemize}
}

\milestone{Q1-Q2 2026}{Phase 4 - Finalisation Nationale}{
    \begin{itemize}
        \item Déploiement sur 100\% du réseau restant
        \item Centralisation supervision nationale
        \item Formation avancée équipes régionales
        \item Documentation complète et transfert connaissances
    \end{itemize}
}

\end{dollaramatimeline}

\clearpage

% ===============================================
% APPENDICES WITH TECHNICAL DETAILS
% ===============================================

\dollaramasectionconclusion{Annexes Techniques}{
    Documentation de support détaillée
}{
    Les annexes suivantes fournissent les détails techniques, financiers et opérationnels nécessaires à l'évaluation complète du projet.
}{images/dollarama_hq_atrium.jpg}

\begin{dollaramaappendices}

\dollaramaappendix[annexe:technical]{Spécifications Techniques Détaillées}
\textbf{Infrastructure Caméras :}
\begin{itemize}
    \item Résolution : 4K (3840×2160) minimum
    \item Technologie : IP avec PoE+ (60W)
    \item Analyse embarquée : IA détection objets et comportements
    \item Vision nocturne : IR 50m minimum
    \item Stockage local : Edge recording 72h minimum
\end{itemize}

\textbf{Plateforme Centrale :}
\begin{itemize}
    \item Architecture : Cloud hybride avec redondance géographique
    \item Capacité : 10,000+ caméras simultanées
    \item Rétention : 90 jours standard, 2 ans événements critiques
    \item API : RESTful pour intégration systèmes tiers
    \item Sécurité : Chiffrement AES-256, authentification multifacteur
\end{itemize}

\dollaramaappendix[annexe:financial]{Modèle Financier Détaillé}
\textbf{Hypothèses Clés :}
\begin{itemize}
    \item Taux d'actualisation : 8\% (WACC Dollarama)
    \item Inflation annuelle : 2.5\%
    \item Réduction pertes progressive : 15\% An1, 20\% An2, 25\% An3+
    \item Durée vie équipements : 7 ans
    \item Valeur résiduelle : 15\% de l'investissement initial
\end{itemize}

\textbf{Analyse Sensibilité :}
\begin{itemize}
    \item Scénario pessimiste (-25\% bénéfices) : ROI 19\%, Payback 26 mois
    \item Scénario optimiste (+25\% bénéfices) : ROI 35\%, Payback 14 mois
    \item Point d'équilibre : Réduction pertes minimum 12\%
\end{itemize}

\dollaramaappendix[annexe:vendors]{Évaluation Fournisseurs}
\textbf{Critères d'Évaluation :}
\begin{itemize}
    \item Expérience retail à grande échelle (20+ points)
    \item Capacités techniques et innovation (25+ points)
    \item Support et maintenance pan-canadien (20+ points)
    \item Proposition financière compétitive (15+ points)
    \item Références clients similaires (10+ points)
    \item Certification sécurité et conformité (10+ points)
\end{itemize}

\textbf{Fournisseurs Préqualifiés :}
\begin{enumerate}
    \item \textbf{Avigilon (Motorola Solutions)} - Score: 92/100
    \item \textbf{Axis Communications} - Score: 88/100
    \item \textbf{Milestone Systems} - Score: 85/100
\end{enumerate}

\end{dollaramaappendices}

\end{document}